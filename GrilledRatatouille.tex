\documentclass [11pt, letterpaper] {article}
\input {headings}
\newcommand \recipeName {Ratatouille Grelhado}
\newcommand \fileName {GrilledRatatouille}
\chead {\recipeName}

\begin {document}
\input {title}

Esta receita foi inspirada por um jantar simples na cal\c{c}ada de um restaurante simples em Lion na Fran\c{c}a.
Esta \'e uma adapta\c{c}\~ao de uma receita para Ratatouille Grelhado Proven\c{c}al do livro {\em The New Professional Chef --- The Culinary Institute of America} editado por Mary Deirdre Donovan, edi\c{c}\~ao 8, 1996. Esta adapta\c{c}\~ao torna a receita muito mais f\'acil de preparar para grandes quantidades para uma festa grande. O melhor resultado se obt\'em preparando o \'oleo de oliva infundido, as cebolas assadas e o molho de tomate dois dias antes da festa. No dia anterior da festa grelhe os vegetais e misture com o molho de tomate. Coloque o Ratatouille no refrigerador durante a noite. No dia de servir, remova do refrigerador cedo para permitir que ele volte a temperatura ambiente. Voce tambem pode aquece-lo gentilmente num forno ligado em 90 C (180 F).
\vspace{0.5in}

\begin{description}

\item[Ingredientes:]\ \\
	\begin{itemize}
	\item 25 dentes de alho (descascados) 
	\item 8 ramos de alecrim
	\item 500 ml de azeite de oliva extra-virgin
	\item 3 cebolas bem grandes (preferencialmente cebolas Vidalia doces, max tambem pose usar cebolas amarelas comuns) 
	\item 2 latas grandes de tomate (tomates italianos de boa qualidade)
	\item 1 vidro de piment\~oes vermelhos grelhados e descascados (ou tres piment\~oes  vermelhos frescos que voce grelha e descasca).
	\item 1 pimenta vermelha forte (no  Brasil, use duas pimentas de cheiro)
	\item Pimenta preta mo\'{\i}da na hora 
	\item Manjeric\~ao fresco (ou salsinha italiana fresca)
	\item Beringelas
	\item Abrobinhas verdes
	\end{itemize}

\item[Procedimento:]\ \\
	\begin{enumerate}
	\item {\bf Azeite perfumado com alecrim e alho} 
	\begin{itemize}
	\item Coloque o azeite de oliva em uma panela rasa.
	\item Corte cinco dentes de alho ao meio e adicione ao azeite.
	\item Adicione os raminhos de alecrim.
	\item Coloque no fogo baixo e aque\c{c}a at\'e estar bem quente, mas antes que o alho esteja fritando.
	\item Deixe esfriar e coloque em um vidro com tampa.
	\item Pode ser armazenado em um local frio por alguns dias at\'e voce estar pronto a continuar a receita. Se colocar no refrigerador, retire do refrigerador algumas horas antes de usar.
	\end{itemize}
	\item {\bf Os pr\'oximos dois passos podem ser feitos dois ou tr\^es dias antes de servir o Ratatouille. }
	\item {\bf Molho de Tomate}
	\begin{itemize}
	\item Ligue o forgo em temperatura m\'edia baixa 150 C (300 F).
	\item Abra as latas de tomate.
	\item Coloque 1/3 de x\'{i}cara do azeite perfumado em uma panela pesada que n\~ao reaja com a acidez do tomate, com tampa e que possa ir no forno.
	\item Corte tr\^es dentes de alho ao meio e adicione \`a panela.
	\item Adicione a pimenta forte picada \`a panela.
	\item Coloque a panela em fogo moderadamente alto e cozinhe someone at\'e o alho ficar com uma cor marrom clara ---- n\~ao cozinhe demais.
	\item Despeje as latas de tomate na panela para reduzir a temperature do alho/azeite e evitar cozinhar o alho demais.
	\item Tempere com um pouquinho de sal e leve o tomate para uma fervura leve.
	\item Coloque a panela em um forno moderado (150 C) e asse, com a panela tampada, por duas horas, misture com uma colher de pau a cada 45 minutes.
	\item Depois de duas horas, adicione os piment\~oes assados cortados em pedacinhos, com o seu suco, \`a panela.
	\item Asse por mais uma hora (remova a tampa da panela se o molho estiver com muita \'agua).
	\item {\bf Nota:} Se estiver usando tomates frescos e maduros, simplesmente lave os tomates, coloque-os inteiros na panela para assar. Antes de adicionar os piment\~oes passe o molho num triturador manual (food mill) com a grelha com os buracos maiores para remover as peles do tomate.
	\end{itemize}
	\item {\bf Cebolas Assadas com Alho}
	\begin{itemize}
        		\item Descasque as cebolas e corte em peda\c{c}os grandes.
		\item Coloque a cebola em um prato que possa ir no forno e que tenha tampa.
		\item Adicione tr\^es colheres de sopa do azeite perfumado.
		\item Adicione os dentes de alho restates (interos).
		\item Tempere levemente com sal.
		\item Se estiver usando cebolas amarelas comuns, adicione 1/2 colher de cha de a\c{c}ucar.
		\item Tempere com pimenta preta mo\'{\i}da na hora.
		\item Misture as cebolas, cubra o prato e coloque no forno a 150 C (300 F) por tr\^es horas, misturando as cebolas a cada 45 minutos - as cebolas e o molho de tomate assam ao mesmo tempo.
		\item As cebolas est\~ao assadas quando elas estiverem moles, o volume estiver reduzido significativamente, e as cebolas est\~ao levemente douradas.
		\item Misture as cebolas assadas no molho de tomato assado.
		\item Deixe esfriar e coloque numa vasilha fechada e coloque no refrigerator.
	\end{itemize}
	\item {\bf Grelhando os as Beringelas}
	\begin{itemize}	
		\item Corte as beringelas ao comprido em fatias grossas (+/- 1 cm).
		\item Polvilhe as fatias com sal nos dois lados (cuidado para n\~ao salgar demais porque elas v\~ao reduzir em volume depois de grelhadas).
		\item Deixe as fatias de beringela salgadas descan\c{c}ar pelo  menos meia hora em um prato fundo ou num escorregador de massa com um prato embaixo. 
		\item Fa\c{c}a o mesmo procedimento com as abobrinhas verdes (fatie, salgue levemente, deixe descansar).
		\item Aque\c{c}a a sua grelha ao m\'aximo.
	 	\item Limpe bem a grelha com uma escova de a\c{c}o.
	 	\item Coloque uma pequena quantidade de \'oleo com alto ponto de queimada (\'oleo de canola, de milho, ou de soja) em uma vasilha pequena, dobre um peda\c{c}o de toalha de papel v\'arias vezes. Segure o papel com pegadores, mergulhe o papel no oleo e passe em cima da grelha. Cubra a grelha e deixe o oleo queimar por um minuto. Repita o processo tr\^es ou quatro vezes para reduzir a possibilidade da comida grudar na grelha.
		\item Reduza a temperatura da grelha para um calor alto moderato.
		\item Seque as fatias de beringela com toalhas de papel.
		\item Coloque um pouco do azeite perfumado em um prato de jantar.
		\item Rapidamente passe os dois lados de cada fatia de beringela no oleo. Continue adicionando mais oleo como necess\'ario. Voce pose passar tantas fatias no azeite quanto caiba na grelha antes de come\c{c}ar a grelhar.		
		\item Grelhe as beringelas ate' que elas estejam macias, vire com uma espatula metalica. 
		\item Quando remover da grelha, coloque em uma tigela e cubra para elas continuaem cozinhando enquanto esfriam.
	\end{itemize}
	\item {\bf Grelhando as Abrobinhas}
	\begin{itemize}			
		\item Corte cada abrobinha ao meio ao comprido.
		\item Corte de novo ao comprido para dividir cada abobrinha em quatro partes longas.
		\item Com uma pequena faca, remova as sementes das abobrinhas.
		\item Salpique cada pedaco de abobrinha levemente com sal e deixe parar um pouco.
		\item Seque as abrobinhas com toalhas de papel.
		\item Coloque um pouco do azeite perfumado em um prato de jantar.
		\item Rapidamente passe os dois lados de cada peda\c{c}o de abrobinha no \'oleo. Continue adicionando mais \'oleo como necess\'ario. Voce pose passar tantas fatias no azeite quanto caiba na grelha antes de come\c{c}ar a grelhar.		
		\item Grelhe as abrobinhas ate' que elas estejam grelhadas mas ainda com uma consist\^encia firme, vire com uma espatula met\'alica. 
		\item Quando remover da grelha, coloque em uma tigela e cubra para elas continuem cozinhando enquanto esfriam.
	\end{itemize}
	\item {\bf Cortando os Vegetais e Misturando no Molho}
	\begin{itemize}	
		\item Quando os vegetais estiverem suficientemente frios para manusear, corte os vegetais grelhados em peda\c{c}os grandes. 
		\item Gentilmente mixture os vegetais, junto com os seus sucos, no molho de tomate e prove o sal.
		\item Coloque de volta no refrigerador e mantenha durante a noite.
	\end{itemize}
	\item{\bf Manjeric\~ao e Servir}
	\begin{itemize}	
		\item No dia que voce for servir, remova da geladeira com v\'arias horas de anteced\^encia para que o ratatouille volte a temperature ambiente.
		\item Pique o manjeric\~ao e misture no ratatouille.
		\item Adicione pimenta preta mo\'{\i}da na hora (a gosto) e prove o sal.
		\item Coloque o  ratatouille em um forgo a 90C at\'e que voce esteja pronto a servir (o prato deve estar apenas levemente aquecido).
	\end{itemize}	
     	\end{enumerate}         
\end{description}
\input{\imageDir/\fileName/imageTable}
\end{document}



