\documentclass [11pt, letterpaper] {article}
\input {headings}
\newcommand \recipeName {Gnocchi}
\chead {\recipeName}

\begin {document}
\input {title}

\begin {flushright}
{\bf From {\it  Italian Table} por Lidia M. Batianich.}
\end {flushright}

\begin {description}

\item [Ingredientes:] \ \\
\begin {itemize}
\item 750 gramas de batatas 
\item 1 ovo grande
\item 1 colher de chá de sal (2)
\item 1/4 colher de chá de pimenta branca moída 
\item Pitada de noz-moscada ralada na hora
\item 1/4 xícara de queijo parmigiano-Reggiano ralado na hora
\item 2 xícaras de farinha, ou a quantia necessária
\item 2 colheres de sopa de salsa italiana picada
\item 4 colheres de sopa de manteiga (opcional para o molho)
\end {itemize}


\item [Procedimento:] \ \\
\begin {enumerate}
\item {\bf Cozinhe as batatas}
	\begin {itemize}
	\item Coloque as batatas em uma panela grande com água fria suficiente para cobrir as batatas.
        \item Leve a água a ferver e cozinhe, parcialmente coberta, até que as batatas sejam facilmente perfuradas com uma faca de ponta, mas as peles não estejam rachando (35 minutos).
\end {itemize}
\item {\bf Esmague as batatas}
	\begin {itemize}
	\item Escorra as batatas e deixe esfriar o suficiente para manusear.
	\item Segurando as batatas com uma toalha de cozinha ou luva, descasque as batatas com uma faca de aparar.
	\item Passe as batatas através de um moinho de comida equipado com o disco fino, deixando-os cair no topo do balcão.
	\item Espalhe as batatas esmagadas em uma camada fina na superfície de trabalho, sem pressioná-las ou compactá-las. Deixe esfriar completamente
	\end {itemize}
\item {\bf Prepare o ovo}
\begin {itemize}
\item Em uma tigela pequena, bata o ovo, o sal, a pimenta e a noz-moscada juntos.
\end {itemize}
\item {\bf Prepare a massa}
\begin {itemize}
\item Espalhe o queijo parmesão ralado sobre as batatas espalhadas e resfriadas.
\item Usando um raspador de balc\~ao junte as batatas e o queijo em um monte com um poço no centro.
\item Despeje a mistura de ovos no poço.
\item Amasse a mistura de batata e ovo adicionando farinha suficiente para fazer uma massa lisa, mas ligeiramente pegajosa. Evite adicionar muita farinha para fazer um nhoque leve.
\item Use um raspador de balc\~ao remova massa de suas mãos e superfície de trabalho enquanto amassa.
\end {itemize}

\item {\bf Forme os nhoques}
\begin {itemize}
\item Crie uma forma retangular com cerca de 3/4 de polegada de espessura no balcão com a massa.
\item Corte a massa em quadrados de 3/4 polegadas.
\item Forme bolinhas grosseiras (voce n\~ao quer manusear a massa muito) com cada quadrado rodando levemente cada uma das suas mãos.
\item Coloque os dentes de um garfo em um ângulo de 45 graus com o balc\~ao com a parte côncava voltada para cima.
\item Pegue cada bola de massa e com a ponta do polegar, pressione a massa ligeiramente contra os dentes do garfo enquanto o rola para baixo em direção à ponta do polegar. À medida que a massa envolve a ponta do polegar, ela se formará em uma bolinha de massa com um recuo profundo de um lado e uma superfície ondulada no outro.
\item Coloque os nhoques em uma assadeira alinhada com uma toalha de cozinha com farol.
\end {itemize}
\item {\bf Pré-Cozinhe o nhoque}
\begin {itemize}
\item Unte uma forma com azeite ou manteiga para colocar o nhoque pré-cozido.
\item Coloque 6 litros de água salgada a ferver em fogo alto at\'e que esteja fervendo rapidamente.
\item Coloque pequenas quantidades de nhoque na água fervente.
\item Cozinhe, mexendo suavemente com uma colher de pau, até ficar macio (cerca de um minuto).
\item Retire o nhoque da água com uma colher furada, tela de aranha ou espumadeira, escorra bem e transfira para a assadeira untada.
\end {itemize}
\item {\bf Refogue o nhoque (com manteiga)}
\begin {itemize}
\item Quando estiver pronto para servir, adicione manteiga a uma frigideira que n\~ao grude, aqueça até que a manteiga adquira uma cor dourada. Tenha cuidado para não queimar a manteiga.
\item Adicione os nhoques pré-cozidos à manteiga dourada e refogue até que comecem a adquirir alguma cor dourada.
\item Polvilhe com salsa picada.
\item Polvilhe com queijo parmesão ralado.
\item Sirva imediatamente.
\end {itemize}
\item {\bf Finalize o nhoque (com sucos de frango assado)}
\begin {itemize}
\item Quando estiver pronto para servir, remova a frango da frigideira em que foi assada.
\item Desnatada parte do excesso de gordura da frigideira, deixando os sucos nele.
\item Traga os sucos e a gordura restante para uma fervura vigorosa na panela do molho.
\item Adicione os nhoques pré-cozidos à manteiga marrom e refogue até que comecem a adquirir alguma cor dourada.
\item Polvilhe com salsa picada.
\item Polvilhe com queijo parmesão ralado.
\item Sirva imediatamente.
\end {itemize}
\end {enumerate}

\end {description}
\end {document}
