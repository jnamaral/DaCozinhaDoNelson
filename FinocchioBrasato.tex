
\documentclass [11pt, letterpaper] {article}
\input {headings}
\newcommand \recipeName {Funcho Refogado}
\chead {\recipeName}

\begin {document}
\input {title}

\begin {flushright}
{\hspace {4in} De Marcella Hazan.}
\end {flushright}
\vspace {0.5in}



\begin {description}

\item [Ingredientes:] \ \\
\begin {itemize}
\item 3 grandes bulbos de funcho ou 4 a 5 menores
\item 1/3 xícara de azeite virgem extra
\item Sal
\end {itemize}

\item [Procedimento:] \ \\

\begin {enumerate}
\item {\bf Prepare o Funcho}
\begin {itemize}
\item Corte os topos do funcho onde eles encontram o bulbo e descate.
\item Remover qualquer parte externas do bulbo que pode ser machucada ou descolorida.
\item Corte 0.5 cm do final da extremidade.
\item Corte o bulbo verticalmente em fatias com cerca de 0.7 cm de espessura.
\end {itemize}

\item {\bf Refogue the Funcho}
\begin {itemize}
\item Coloque o funcho e o azeite em uma panela grande.
\item Polvilhe com sal.
\item Adicionar água suficiente para cobrir mal a erva-doce.
\item Conecte o calor ao meio.
\item Não tampe a panela.
\item Cozinhe, virando as fatias de vez em quando até que o funcho esteja  dourado e macio quando perfurado com uma faca de ponta - entre 25 e 40 minutos.
\item Se o líquido é insuficiente, adicione um pouquinho mais de água.
\item Toda a água deve ser evaporada o funcho est\'a cozido.
\item Sirva em uma prato quente.
\end {itemize}

\end {enumerate}
\end {description}
\end {document}
