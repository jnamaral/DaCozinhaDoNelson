\documentclass [11pt, letterpaper] {article}
\input {headings}
\newcommand \recipeName {Recheio de Frango}
\chead {\recipeName}

\begin {document}
\input {title}


Este \'e um recheio de frango que pode ser usado em muitas receitas incluindo pastel\~ao, risolis, past\'eis. Tamb\'em \'e delicioso como um simples sandu\'iche prensado ou em pizza. Eu recebi esta receita da minha m\~ae e eu sei que ela usa freq\"uentemente.

\begin {description}

\item [Ingredientes:] \ \\
\begin {itemize}
\item 1 frango
\item 2 dentes de alho
\item 1/2 colheres de sopa de sal
\item 1/4 colher de chá de pimenta preta recém-castrada
        \item 2 colheres de sopa de óleo de cozinha sem sabor (como canola ou girassol)
        \item 2 xícara de água
        \item 1 cebola média
        \item 1 colher de chá de mel
\item 1 lata de 450 gramas de tomates italianos
\item 3 ovos cozidos perfeitos (veja a \href{HardBoiledEggs.html}{Receita Mestre de Julia Child})
        \item salsinha italiana
        \item pimenta do reino mo\'ida na hora
\end {itemize}


\item [Procedimento:] \ \\
	\begin{enumerate}
	\item{\bf Tempere o frango}
	\begin{itemize}
        \item Corte a galinha (alternativamente use apenas sobrecoxas de frango).
        \item Polvilhe galinha com sal.
        \item Coloque o frango em um recipiente fechado e coloque-o
              o refrigerador por pelo menos 1 hora (até 12 horas).
        \end{itemize}
 	\item{\bf Dore e cozinhe of frango}
	\begin{itemize}       
        \item Aquecer o óleo de cozinha em uma panela salteada de fundo pesado em fogo moderadamente alto.
        \item Dore brandamente a galinha em todos os lados.
        \item Descasque os dentes de alho, corte ao meio e adicione \`a panela.
        \item Adicione 1 xícara de água, leve a ferver, reduza o fogo
          e misture até que a galinha esteja macia (remova o peito mais cedo para que não fique resequido).
        \item Depois de remover o peito, continue cozinhando até que toda a água tenha evaporado. Somente os pedaços de frango escuro devem estar na panela.
        \item Remova os pedaços restantes de frango para um prato para esfriar.
        \end{itemize}
 	\item{\bf Fa\c{c}a o molho}
	\begin{itemize}
        \item Remova toda a gordura para um pequeno recipiente e reserve.
        \item Adicione 1 xícara de água aos sólidos na panela, raspe e deixe esfriar até ter um caldo levemente engrossado na panela.
        \item Remova todo o caldo da panela com uma espátula de borracha,
        coloque em um pequeno recipiente e reserve.
        \item Corte a cebola bem miudinha.
        \item Adicione duas colheres de sopa da gordura reservada à panela e aque\c{c}a em fogo moderadamente alto.
        \item Adicione a cebola picada e refogue por cinco minutos até a cebola começar a dorar.
        \item Adicione 1 colher de chá de mel \`a cebola e continue refogando até ficar bem dourada.
        \item Adicione a lata de tomates e o caldo de galinha reservado.
        \item Deixe cozinhar até que a maioria da água tenha evaporado e a gordura se separe do molho de tomate.
         \end{itemize}
 	\item{\bf Desosse e pique o frango}
	\begin{itemize}       
        \item Enquanto isso, remova a carne da frango j\'a fria. Descarte os ossos, a as veias. 
        \item Corte a pele dourada e cozida em um guizadinho muito fino.
        \item Corte a carne de frango em peda\c{c}os muito pequenos.
         \end{itemize}
 	\item{\bf Termine o recheio}
	\begin{itemize}      
	\item Quando o molho de tomate estiver reduzido, adicione o frango em cubos e cozinhe por mais alguns minutos mexendo.
	\item Se for usar o recheio dentro de uma massa, como o pastel\~ao ou os risolis, misture uma colher de sopa de amido de milho com 1/4 de x\'icara de agua fria at\'e dissolver o amido. Despeje a mistura de amido e \'agua no recheio quente e misture bem at\'e engrossar um pouco (um minuto).
        \item Despeje em uma tigela de vidro ou metal, deixe esfriar completamente (pode ser refrigerado por até três dias antes do uso, também pode ser congelado e descongelado antes de usar).
        \item O recheio pode ser usado assim como est\'a para muitas aplica\c{c}\~oes. Ele tamb\'em pode ser temperado com uma erva fresca. Sugest\~oes incluem basilico, s\'alvia, alecrim, oregano, ou cebolinha verde fininha. Eu sugiro que use apenas uma erva de cada vez e apenas em uma quantidade moderada.
         \end{itemize}
        \item {\bf Adicione os ovos cozidos (logo antes de usar)}
	\begin{itemize} 
	\item Ferva 3 ovos usando a receita mestre de Julia Child para ovos perfeitos cozidos de ``A maneira de cozinhar".
        \item Separe as claras das gemas cozidas.
        \item Corte as claras em peda\c{c}os muito pequenos e adicione ao recheio frio.
        \item Corte as gemas  em peda\c{c}os muito pequenos e adicione ao recheio frio.
        \item Pique uma salsinha italiana muito finamente e adicione ao recheio.
        \item Experimente e corrija a pimenta preta no recheio. Mexa bem.
         \end {itemize}
\end {enumerate}
\end {description}

\begin{table}
\begin{tabular}{cccc}
\includegraphics[width=0.25\textwidth]{\imageDir/ChickenFilling/IMG_8520.jpg} &
\includegraphics[width=0.25\textwidth]{\imageDir/ChickenFilling/IMG_8521.jpg} &
\includegraphics[width=0.25\textwidth]{\imageDir/ChickenFilling/IMG_8522.jpg} &
\includegraphics[width=0.25\textwidth]{\imageDir/ChickenFilling/IMG_8523.jpg} \\
\includegraphics[width=0.25\textwidth]{\imageDir/ChickenFilling/IMG_8524.jpg} &
\includegraphics[width=0.25\textwidth]{\imageDir/ChickenFilling/IMG_8525.jpg} &
\includegraphics[width=0.25\textwidth]{\imageDir/ChickenFilling/IMG_8526.jpg} &
\includegraphics[width=0.25\textwidth]{\imageDir/ChickenFilling/IMG_8527.jpg} \\
\includegraphics[width=0.25\textwidth]{\imageDir/ChickenFilling/IMG_8528.jpg} &
\includegraphics[width=0.25\textwidth]{\imageDir/ChickenFilling/IMG_8529.jpg} \\
\end{tabular}
\end{table}

\end {document}
