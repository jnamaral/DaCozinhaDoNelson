\documentclass [11pt, papel de carta] {article}
\input {headings}
\newcommand \recipeName {Porquinhos em um Cobertor}
\newcommand \fileName {PigsOnBlanket}
\chead {\recipeName}

\begin {document}
\input {title}


Este \'e um salgado feito de salsichas enroladas em massa de p\~ao. Em ingl\^es estes salgados s\~ao chamados de ``Pigs in a Blanket" porque a salsicha \'e feita de carne de porco. Para a massa, eu uso a receita para o ``Melhor P\~aozinho Americano para Jantar" da {\it Cook's Illustrated}. Para uma versão sem leite, substitua o leite por leite de amêndoa e a manteiga por margarina de assar. Esta receita é muito conveniente para uma festa, porque a massa é preparada e os porquinhos s\~ao enrolados no cobertor até dois dias antes de assar e servir. Devem ser removidos da geladeira para crescer em um lugar com temperatura fresca por 6 a 7 horas antes de assar.

\vspace {0.3in}
\begin {description}

\item [Ingredientes:] \ \\
\begin {itemize}
\item 3/4 xícara de leite integral
\item 6 colheres de sopa de manteiga sem sal, derretida
\item 6 colheres de sopa de açúcar
\item 1 1/2 colheres de chá de sal de mesa
\item 2 ovos grandes, a temperatura ambiente
\item 1 pacote de fermento rápida (2 1/4 colheres de chá), também pode ser rotulado como "instantâneo"
\item 3 xícaras de farinha (425 gramas)
\item salsichas
\item 2 colheres de sopa de manteiga sem sal, derretida (para escovar os porquinhos antes de assar)
\end {itemize}

\item [Procedimento:] \ \\
\begin {enumerate}
\item {\bf Faça a base da massa}
\begin {itemize}
\item Ferva o leite em uma panela pequena em fogo médio; Deixe esfriar até que uma pele se forme na superfície,
3 a 5 minutos. Usando uma colher de sopa, remova a pele da superfície e descarte.
\item Transfira o leite para o prato da batedeira e adicione 6
colheres de sopa de manteiga derretida, açúcar e sal.
\item Misture com um bal\~ao para combinar e deixe a mistura esfriar.
\item Quando a mistura estiver apenas quente ao toque (32 a 38 graus C), bata os ovos e o fermento até combinar.
\end {itemize}


\item {\bf Adicione a farinha}
\begin {itemize}
\item Adicionar farinha à tigela; usando o gancho de massa, misture em baixa velocidade na batedeira até combinar, 1 a 2 minutos.
\item Aumentar a velocidade para média-baixa e processar na batedeira por cerca de mais 3 minutos. Quando pressionada com o dedo, a massa deve sentir-se pegajosa e úmida, mas não deve grudar
no dedo. (Se a massa estiver muito pegajosa, adicione uma colher de farinha de cada vez at\'e que a massa n\~ao esteja t\~ao pegajosa). 
\item Continue a amassar em velocidade média-baixa até que uma massa coesa e elástica
se forme (deve desgrudar dos lados da tigela da batedeira, mas ainda grudar no fundo), 4 a 5 minutos por mais.
\end {itemize}

\item {\bf Amasse a massa}
\begin {itemize}
\item Transfira a massa para superfície de trabalho levemente enfarinhada.
\item Amasse manualmente por 1 a 2 minutos para garantir que esteja bem amassada. A massa
deve ser muito macia e úmido, mas não excessivamente pegajosa. (Se a massa se encaixar excessivamente nas mãos e na superfície de trabalho, adicione uma colher de sopa de farinha de cada vez até que a massa possa ser trabalhada.)
\item Pulverize levemente uma tigela média com um spray para cozinhar antiaderente.
\item Transfira a massa para a tigela.
\item Cubre levemente a superfície da massa com spray de cozinha e cubra com uma película de plástico.
\item Deixe a massa crescer em um lugar morno e sem vento, até dobrar em volume, 2 a 3 horas.
\end {itemize}

\item {\bf Preparar as formas}
\begin {itemize}
\item Unte duas formas com oleo.
\item Coloque duas colheres de sopa de farinha em um  pequeno coador e use-o para pulverizar as formas.
\item Bata a assadeira no balcão várias vezes até ficar uniformemente revestida com a farinha.
\item Despeje qualquer excesso de farinha excessiva na pia.
\end {itemize}

\item {\bf Forme os porquinhos}
\begin {itemize}
\item Role a massa em um retangulo de 45 cm x 60 cm e apare as bordas.
\item Corte as salsichas em tr\^es peda\c{c}os iguais. As salsichas grandes tem 16.5 cm de comprimento, cada ter\c{c}o ter\'a 5.5 cm.
\item Usando um peda\c{c}o de cachorro-quente como guia, fa\c{c}a v\'arios cortes pequenos para marcar uma tira de massa que \'e larga o suficiente para rolar os peda\c{c}os de cachorros quentes com o suficiente nas extremidades para fechar bem as pontas.
\item Usando os cortes pequenos como guia, corte as tiras.
\item Role cada cachorro quente cortando a tira de massa para fazer uma costura fechada.
\item Pressione ao longo da emenda e coloque o lado da emenda para baixo nas bandejas preparadas.
\end {itemize}

\item {\bf Refrigere por um dia ou dois}
\begin {itemize}
\item Cubra as formas com papel de plástico levemente untado com oleo.
\item Cubra as formas de maneira segura com papel alumínio.
\item Refrigere pelo menos 24 ou até 48 horas.
\end {itemize}

\item {\bf Deixe os porquinhos crescerem}
\begin {itemize}
\item Remova a forma (mas não o plástico) das formas.
\item Deixe os porquinhos crescerem em temperatura ambiente fria sem ventos
 até dobrar em volume, 6 a 7 horas.
\item Quando os porquinhos tiverem quase dobrados em volume,
ajuste a grade do forno para a posição do meio-baixo e aqueça o forno a 200 C.
\end {itemize}

\item {\bf Asse os porcos}
\begin {itemize}
\item Remova a embalagem plástica.
\item Pinte os porquinhos com 2 colheres de sopa de manteiga derretida;
\item Asse até ficar dourado, 14 a 18 minutos. Gire as formas na metade do tempo para garantir um cozimento uniforme.
\item Deixe of porquinhos esfriarem um pouquinho nas formas em uma grade por 3 minutos.
\item Mova os porquinhos para uma grade e deixe esfriar por mais 10 a 15 minutos.
\item Sirve morno.
\end {itemize}
\end {enumerate}
\end {description}

\input{\imageDir/\fileName/imageTable}

\end {document}
