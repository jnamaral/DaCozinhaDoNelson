\documentclass [11pt, letterpaper] {article}
\input {headings}
\newcommand \recipeName {Massa de Pizza Macia}
\chead {\recipeName}

\begin {document}
\input {title}

Em casa, preferimos a massa de pizza macia que é mais típica da pizza profunda estilo Chicago. Nesta receita, eu substitui a manteiga por gordura vegetal. Se você não tem uma restrição de leite, você pode usar manteiga em vez da gordura vegetal.

\vspace {0.3in}

\begin {description}

\item [Ingredientes:] \ \\
\begin {itemize}
\item 10 onças de água
\item 1 colher de chá de açúcar
\item 1/2 colher de chá de sal
\item 2 colheres de sopas de verdura vegetal,
\item 2 1/2 xícaras de farinha (350 grams)
\item 1 colher de sopa de fermento
\end {itemize}

\item [Procedimento:] \ \\

\begin {enumerate}
\item {\bf Preparando a massa}
\begin {itemize}
\item Se estiver usando a máquina de pão, coloque todos os ingredientes na forma da m\'aquina e coloque-o no ciclo de fazer massa (dough cycle).
\item Se estiver usando uma batedeira, coloque a \'agua, o açúcar, e o sal na tigela.
\item Derreta a gordura vegetal no microondas, acrescente ao leite e misture.
\item Adicione o fermento e a farinha e misture usando a pá at\'e você obter uma massa macia.
\item Remova a pá, cubra a tigela com pl\'astico e deixe crescer por 1 hora a 1,5 horas ou até que tenha dobrado em volume.
\end {itemize}

\item {\bf Preparando a massa para a pizza}
\begin {itemize}
\item Esta é uma massa muito molhada, assim você pode precisar incorporar farinha adicional agora para formar pizzas individuais.
\item Coloque a massa em uma superfície enfarinhada e perfure a massa para eliminar bolhas de ar e amassar. Tenha cuidado para não trabalhar a massa demais.
\item Forme as pizzas e deixe-as crescer por pelo menos 30 minutos antes do cozimento.
\item Alternativamente, você pode pressionar a em uma panela que tenha sido untada com azeite --- use oleo nas suas m\~aos tamb\'em ---  e deixe crescer na panela.
\item Se estiver fazendo uma pizza na assadeira, pre-aque\c{c}a o forno a 170 C e asse a massa sem cobertura por 20 minutos. Retire do forno, coloque a cobertura, e retorne ao forno.
\end {itemize}

\end {enumerate}
\end {description}
\end {document}
