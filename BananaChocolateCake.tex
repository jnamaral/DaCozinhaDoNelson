\documentclass [11pt, letterpaper] {article}
\input {headings}
\newcommand \recipeName {Bolo de Banana e Chocolate do Daniel}
\chead {\recipeName}

\begin {document}
\input {title}

\begin {flushright}
Adaptado do livro {\it Better Homes and Gardens Cookbook}
\end {flushright}

Esta \'e minha adapta\c{c}\~ao de uma receita de bolo de banana de um livro Americano tradicional de receitas. Esta vers\~ao n\~ao tem leite e tem chocolate. 
\ \newline
\begin{description}

\item[Ingredientes:]\ \\
	\begin{itemize}
	\item 85 gramas de chocolate amargo (70\% de cacau) de boa qualidade
	\item 1/2 x\'icara de gordura vegetal hidrogenada
	\item 1 1/2 x\'icara (315 gramas) de a\c{c}ucar
	\item 1 x\'icara de bananas maduras esmagadas (3 bananas grandes ou 5 pequenas)
	\item 2 ovos
	\item 1 colherinha de ch\'a de baunilha
	\item 1/2 colherinha de ch\'a de sal 
	\item 3/4 x\'icara (180 grams) de leite de amendoa 
	\item 1 colher de sopa de suco de lim\~ao 
        \item 2 1/4 x\'icaras (330 gramas) de farinha
	\item 1 1/2 colherinha de ch\'a de fermento qu\'{\i}mico
	\item 1 colherinha de ch\'a de bicarbonato de s\'odio
	\end{itemize}
	
\item[Ingredientes (cobertura opcional de chocolate):]\ \\
	\begin{itemize}
	\item 120 gramas de chocolate amargo (70\% de cacau)
	\item 4 colheres de sopa de margarina 
	\item 2 colheres de sopa de xarope de milho (karo) 
	\item 1 colherinha de ch\'a de baunilha
	\end{itemize}
	
\item[Procedimento:]\ \\
	\begin{enumerate}
	\item {\bf Prepare uma forma com furo no meio e aque\c{c}a o forno}
	\begin{itemize}
	\item Unte a forma com gordura vegetal, e ent\~ao polvilhe com farinha ou chocolate em p\'o. 
	\item Aque\c{c}a o forno a 175 graus Celsius
	 \end{itemize}
	\item {\bf Prepare a Mistura de Chocolate}
	\begin{itemize}
	\item Derreta o chocolate e a gordura vegetal num prato fundo grande de vidro ou de plastico que possa ir no microondas --- melhor e' usar um prato em que voc\^e possa colocar todos os ingredientes depois.
	 \item Adicione o a\c{c}ucar e misture com um chicote
	 \item Adicione as bananas esmagadas e misture 
	 \item Adicione os ovos e misture bem 
	 \item Adicione a baunilha e o sal
	\end{itemize}
	\item {\bf Substituto para o soro de leite}
	\begin{itemize}
        		\item Em um copo de medida misture o leite de amendoa e o caldo de lim\~ao (o leite pode separar). 
	\end{itemize}
	\item {\bf A mistura seca}
	\begin{itemize}
	\item Em uma outra vasilha misture a farinha, o fermento qu\'{\i}mico e o bicarbonato de s\'odio
       	\end{itemize}
	\item {\bf Complete a mistura}
	\begin{itemize}
        \item Adicione 1/3 da mistura de farinha \`a mistura de chocolate e misture at\'e estar homog\^enea.
        \item Adicione 1/3 da mistura de leite e misture at\'e estar homog\^enea. 
        \item Repita duas vezes mais at\'e que toda a farinha e todo o leite estejam incorporados.
        \end{itemize}
        \item {\bf Asse o bolo}
	\begin{itemize}
        \item Coloque a massa do bolo na forma preparada 
        \item Asse por 40 minutos (ou por mais tempo) at\'e que um palito inserido no centro do bolo saia limpo.
        \end{itemize}
        \item {\bf Esfrie e desinforme o bolo}
        \begin{itemize}
        \item Deixe o bolo esfriar at\'e que esteja morno. 
        \item Passe uma faca ao redor da forma e do centro para soltar o bolo. 
        \item Inverta o bolo em uma t\'abua de cortar limpa (o bolo vai ficar virado para baixo).
        \item Coloque um prato no fundo do bolo e inverta no prato (se voc\^e planeja colocar a cobertura inverta o bolo em uma grade de metal e coloque a grade em cima de uma forma de assar grande. 
         \end{itemize}
        
         \item {\bf Cobertura (opcional)}
       	\begin{itemize}
        \item Em um prato fundo em cima de uma panela de \'agua fervendo lentamente, derreta o chocolate com a margarina ou gordura vegetal e misture at\'e ficar bem homog\^eneo. Remova o prato do fogo, adicione o karo e a baunilha, e misture at\'e estar bem homog\^eneo e brilhante.
        \item Espalhe a cobertura sobre o bolo e deixe esfriar.
         \end{itemize}
     	\end{enumerate}         
\end{description}
\end{document}



