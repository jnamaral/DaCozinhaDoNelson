\documentclass [11pt, papel de carta] {article}
\input {headings}
\newcommand \recipeName {Peixe com Chermoula de Essaouira}
\newcommand \fileName {FishWithEssaouiraChermoula}
\chead {\recipeName}

\begin {document}
\input {title}

Chermoula é e' um tempero composto de ervas e especiarias e limão que os marroquinos costumam usar no peixe. Existem muitas versões de Chermoula. Esta é da cidade de Essaouira. Minha modificação é adicionar Pimenta de Cheiro, uma pimenta perfumada e não tão quente do Brasil. Várias variacões de Chermoulas mencionam o uso de uma pimenta que não é muito quente. Por coincidencia, quando eu estava pesquisando esta receita, eu li uma nota na Cook's Illustrated (setembro / outubro de 2018, página 5) sobre como uma enzima no alho, alliinaze, que converte um composto químico no alho, alliin, em outro composto pungente, alicina, em apenas trinta segundos depois do alho ser cortado. A adição rápida de um ácido ao alho evita essa reação, mantendo o alho suave. Você quer o sabor mais suave de alho neste prato de peixe. Assim, a velocidade é importante ao misturar o alho com o limão.

\begin {description}

\item [Ingredientes:] \ \\
\begin {itemize}
\item 1 colher de chá de cominho
\item 1/2 colher de chá de coentro
\item 1 colher de chá de páprica
\item 2 pimenta de cheiro
\item 1 colher de chá de sal ou a gosto
\item 3 dentes de alho descascados
\item 2 colheres de sopa de suco de limão espremido na hora
\item 4 colheres de sopa de salsa de folhas planas fresca picada
\item 4 colheres de sopa de azeite
\item 4 pedaços de filé de bacalhau
\end {itemize}


\item [Procedimento:] \ \\
\begin {enumerate}
\item {\bf Torre e moa as especiarias:}
\begin {itemize}
\item Corte um retângulo relativamente pequeno de papel-manteiga, faça uma pasta no papel e coloque-a no balcão.
\item Coloque cominho e coentro em uma frigideira seca.
\item Coloque a frigideira em fogo moderado e torre as sementes até que fiquem cheirosas.
\item Imediatamente transfira as especiarias torradas para o papel.
\item Use o papel para despejar as especiarias torradas em um moedor de café ou moedor de especiarias.
\item Moa as especiarias em um pó.
\item Se tiver pedacos grande de casca do coentro, passe as especiarias moidas em um coador.
\end {itemize}
\item {\bf Prepare o Chermoula:}
\begin {itemize}
\item Esprema o suco de limão coando as sementes em uma tigela pequena.
\item Coloque o alho na tábua de corte.
\item Esmague o alho com uma faca de chef.
\item Agindo rapidamente, adicione o sal ao alho e esmague em uma pasta.
\item Imediatamente transfira a pasta de alho para a tigela pequena de suco de limão e use uma colher pequena para mexer bem o mais rápido possível.
\item Corte a pimenta de cheiro em uma polpa e adicione na tigela
\item Corte a salsinha em uma costeleta bem fina e adicione à tigela.
\item Despeje as especiarias moídas e a páprica na tigela.
\item Mexa bem
\end {itemize}
\item {\bf Marinar o peixe}
\begin {itemize}
\item Cerca de duas horas antes de cozinhar seque os filés de peixe frios com toalhas de papel.
\item Esfregue o Chermoula no peixe.
\item Cubra o peixe com uma folha e deixe descansar, à temperatura ambiente, por duas horas.
\end {itemize}
\item {\bf Refogue o peixe}
\begin {itemize}
\item Aqueça uma travessa no forno ou com água quente.
\item Aqueça uma frigideira antiaderente até ficar bem quente e despeje azeite suficiente para cobrir apenas o fundo.
\item Refogue o peixe certificando-se de que esteja mal cozido. Se for fazer uma grande quantidade, é especialmente importante para o cozimento do primeiro lote ser mal cozido pois o peixe continua cozinhando depois que ele é removido da panela e outras peças quentes são colocadas em cima.
\item Assim que cada peça estiver pronta, retire e coloque  na travessa quente (em temperatura ambiente) e cubra de maneira solta com papel aluminio.
\item Despeje o molho que formou na panela ou restante azeite na panela sobre o peixe.
\item Sirva imediatamente
\end {itemize}
\end {enumerate}
\end {description}
\input{\imageDir/\fileName/imageTable}
\end {document}
