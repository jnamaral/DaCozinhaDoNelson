\documentclass [11pt, letterpaper] {article}
\input {headings}
\newcommand \recipeName {Salada de Batata da Suábia}
\chead {\recipeName}

\begin {document}
\input {title}

No verão de 2017, n\'os viajamos com minha mãe pelo sul da
Alemanha. Foi uma viagem para visitar os lugares de onde nossos
antepassados alemães partiram para ir ao Brasil em 1828. Um domingo à
tarde, fomos visitar um Biergarten com música clássica fora de
Munique chamado Waldwirtschaff. Lá eu provei uma salada de batata alemã que criou uma memória
de paladar duradoura. Voltando para casa comecei a pesquisar e
descobri que era o que se chama de salada de batata da Suábia. A Su\'abia
é uma região no sudoeste da Alemanha. Tive a sorte de encontrar esta
receita para \href
{https://www.daringgourmet.com/restaurant-style-schwabischer-kartoffelsalat-swabian-potato-salad/}
{Salada de batata Su\'abia em um blog da Kimberly}. O blog contém
imagens passo a passo que eu recomendo rever.

Esta salada \'e excelente para uma festa pois ela \'e ainda mais
gostosa no dia seguinte e portanto pode ser feita um dia antes.
 
\begin {description}

\item [Ingredientes:] \ \\
\begin {itemize}
\item 1 1/2 kgs de batatas amarrela pequenas e de tamanho similar,
peles esfregadas mas com casca
\item 1 cebola amarela média, picada
\item 1 1/2 xícaras de água misturada com 4 colheres de chá de grânulos de caldo de carne (Veganos: usem caldo vegetal)
\item 1/2 xícara de vinagre branco (adicione algumas gotas de Essig
Essenz se você o tiver. O Essig Essenz \'e um vinagre em alta concentra\c{c}\~ao)
\item 3/4 colher de sopa de sal
\item 3/4 colher de chá de pimenta branca moída
\item 1 colher de chá de açúcar
\item 2 colheres de chá de mostarda alemã leve (recomendo a Mostarda alemã de estilo Düsseldorf. Se não conseguir, use mostarda amarela normal)
\item 1/3 copo de óleo de prova neutra
\item Cebolinha picada fresca para enfeitar
\end {itemize}
\ \\

\item [Procedimento:] \ \\
\begin {enumerate}
\item {\bf Ferva e corte as batatas}
\begin {itemize}
\item Ferva as batatas nas suas peles em água levemente salgada até ficarem macias.
\item Deixe as batatas esfriarem até que você possa lidar com elas.
\item Descasque as batatas e corte-as em fatias de 6 mm de grossura.
\item Coloque as batatas cortadas em uma grande tigela e reserve.
\end {itemize}
\item {\bf Tempere as batatas}
\begin {itemize}
\item Adicione a cebola picada, caldo de carne, vinagre, sal, pimenta, açúcar e mostarda em uma panela média e levar a ferver.
\item Assim que ferver, retire do fogo e despeje a mistura sobre as batatas.
\item Cubra a tigela de batatas e deixe-se sentar por pelo menos uma hora.
\end {itemize}
\item {\bf Terminar e servir}
\begin {itemize}
\item Depois de pelo menos uma hora, gentilmente misture o óleo vegetal e tempere com sal e pimenta a gosto.
\item Se houver muito líquido, use uma colher com furos para servir.
\item Sirva decorada com cebolinha verde picada fresca. Sirva a temperatura ambiente.
\item Esta salada de batata é melhor no dia seguinte (remova do frigerador pelo menos 30 minutos antes de servir).
\end {itemize}
\end{enumerate}
\end {description}
\end {document}
