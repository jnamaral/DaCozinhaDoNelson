\documentclass [11pt, letterpaper] {article}
\input {headings}
\newcommand \recipeName {Bolo de Ma\c{c}\~a do Daniel}
\chead {\recipeName}

\begin {document}
\input {title}

Esta e' uma torta popular, a minha adapta\c{c}\~ao foi remover as nozes, adicionar a chimia de ma\c{c}a ou a banana amassada e reduzir o a{c}cucar. A chimia de ma\c{c}a e' chamada de {\it apple sauce} e consiste simplesmente de ma\c{c}a cozida e amassada sem nenhuma adi\c{c}\~ao de a\c{c}ucar.

\begin {description}
\item [Ingredientes molhados:] \ \newline
\begin {itemize}
\item 1/2 xícara de chimia de maçã sem ac{c}ucar (ou duas bananas amassadas)
\item 1 1/2 xícara de açúcar (300 gramas)
\item 3/4 xícara de óleo de canola (160 gramas)
\item 1/4 colher de chá de sal
\item 1 colher de chá de baunilha
\end {itemize}

\item [Ingredientes secos:] \ \newline
\begin {itemize}
\item 3 xícaras de farinha de trigo (15 oz)
\item 1 colher de chá de canela 
\item 1 colher de chá de bicarbonato de sódio
\item 1 colher de chá de fermento em pó
\item 3 xícaras de maçãs em cubos
\item 1/2 xícara de passas de uva
\end {itemize}

\item [Preparação:] \ \newline
\begin {enumerate}
\item Pré-aqueça o forno a 175 C.
\item Prepare uma panela de tubo (unte com margarina ou \'oleo e pulverize com farinha).
\item Em uma tigela, misture os ingredientes secos: farinha, fermento em p\'o, canela,
e bicarbonato de sódio.
\item Em uma tigela grande, misture as bananas (ou despeje o molho de maçã), misture
açúcar, sal, óleo e baunilha.
\item Misture lentamente metade dos ingredientes secos com os molhados.
\item Misture as maçãs cortadas em pedaços e passas com a outra metade dos ingredientes secos at\'e as frutas ficarem cheias de farinha.
\item Despeje as frutas e o restante da farinha que n\~ao tiver incorporado \`a massa e misture at\'e n\~ao aparecer farinha seca.
\item Despeje na forma preparada.
\item Asse por cerca de 1 hora e 10 minutos, até que um palito inserido
no centro do bolo saia limpo.
\item Deixe esfriar um pouco em uma grade de metal antes de retirar da panela.
\end {enumerate}
\end {description}
\end {document}
