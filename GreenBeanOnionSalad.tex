\documentclass [11pt, letterpaper] {article}
\input {headings}
\newcommand \recipeName {Salada de Vagens e Cebola}
\chead {\recipeName}

\begin {document}
\input {title}

\begin {description}

\item [Ingredientes:] \ \\
\begin {itemize}
\item 400  gramas de vagens
\item 2 \href{PickledRedOnions.html}{cebolas vermelhas em conserva}
\item 1 xícara de vinagre branco destilado
\item 1/3 xícara de açúcar
\item 1/4 colher de chá de sal
\item 1 colheres de sopa de sementes de erva-doce
\item Azeite
\item Pimenta preta mo\'ida na hora
\item Vinagre de arroz
\item Manjericão fresco
\item 1/2 colher de chá de sal (ou a gosto)
\end {itemize}

\item [Procedimento:] \ \\
\begin {enumerate}
\item {\bf Pique e Cozinhe as Vagens}
\begin {itemize}
\item Corte as pontas de cada vagen e jogue fora.
\item Corte as vagens na diagonal em fatias bem fininhas. 
\item Coloque uma panela  grande de água a ferver at\'e que atinja uma fervura rápida.
\item Encha uma bacia grande com água fria e adicione bastante gelo à bacia.
\item Coloque a vagem picada na água fervente e cubra imediatamente para que a \`agua volte a ferver rapidamente.
\item Deixe cozinhar por cerca de dois minutos.
\item Teste um pedaço de vagem. Deve estar muito firme mas n\~ao deve estar rijo.
\item Retire as vagens da água fervente com uma espumadeira (o ideal \'e um coador de arame usado pelos chineses).
\item Abaixe-os na tigela de água fria.
\item Deixe esfriar completamente na água fria.
\end {itemize}
\item {\bf Assemble the Salad}
\begin {itemize}
\item Coloque as vagens em uma tigela grande.
\item Tempere com azeite, vinagre de arroz, sal e pimenta a gosto.
\item Em uma frigideira seca aqueça as sementes de erva-doce até ficarem muito perfumadas.
\item Remova para uma plato e deixe esfriar completamente.
\item Adicione cebolas escorridas e sementes de erva-doce \`as vagens.
\item Adicione o manjeric\~ao na hora de servir.
\end {itemize}
\end {enumerate}
\end {description}
\end {document}
