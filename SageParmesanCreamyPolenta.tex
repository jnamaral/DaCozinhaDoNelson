\documentclass[11pt,letterpaper]{article}
%%\documentclass[12pt]{article}

\usepackage{fancyhdr}
\usepackage{fullpage}

\setlength{\topmargin}{0.30in}
\setlength{\headheight}{50pt}
\renewcommand{\baselinestretch}{0.8}
\chead{Polenta Cremosa com S\'alvia e Parmes\~ao}
\lhead{Jos\'{e} Nelson Amaral}
\rhead{jamaral@ualberta.ca}
\cfoot{}
\lfoot{}
\rfoot{}
\pagestyle{fancy}
\renewcommand \refname {Polenta Cremosa com S\'alvia e Parmes\~ao}

\date{}
\begin{document}
\setcounter{page}{1}

\vspace{2.0in}
\ \newline
\ \newline
\begin{centering}
{\bf \Large Polenta Cremosa com S\'alvia, Parmes\~ao e Manteiga Tostada}\\
\end{centering}
\vspace{0.5in}

Eu desenvolvi esta receita inspirado em Gnocchi a la Romana. Ela tem sabores similares ao gnocchi, mas tem uma conist\^encia cremosa para os convidados servirem de colheradas. Portanto \'e mais f\'acil de preparar e n\~ao precisa cortar em pequenos peda\c{c}os e assar como o gnocchi.
 
\begin{description}

\item[Ingredientes:]\ \\
	\begin{itemize}
	\item 5 a 6 x\'{\i}caras de leite integral
	\item 1 grande ma\c{c}o de s\'alvia fresca dividido em duas metades
	\item 1 x\'{\i}cara de polenta ou farinha de milho
	\item 6 colheres de manteiga
	\item 3/4 colherinha de ch\'a de sal
	\item 1 x\'{\i}cara de queijo parmes\~ao ralado na hora 
	\item Pimenta do reino, de pref\^encia mo\'{\i}da na hora
	\end{itemize}

\item[Procedimento:]\ \\
	\begin{enumerate}
	\item {\bf Infundir o Leite com a S\'alvia}
	\begin{itemize}
	\item Coloque 4 x\'{\i}caras de leite em uma panela com caldo grosso.
	\item Divida o ma\c{c}o de s\'alvia em duas metades. Coloque uma metade no leite.
	\item Adicione o sal ao leite.
	\item Ferva o leite com a s\'alvia.
	\item Desligue o fogo, cubra a panela e deixe parado por 10 a 15 minutos.
	\end{itemize}
	\item {\bf Cozinhe a Polenta}
	\begin{itemize}
	\item Remova a tampa, remova a s\'alvia do leite e jogue fora.
	 \item Leve o leite a fervura de novo.
	 \item Lentamente despeje a polenta no leite fervendo mexendo constantemente com um bal\~ao para n\~ao embolar.
	 \item Continue mexendo enquanto a mistura engrossa (em torno de 3 minutos).
	 \item Reduza o fogo para a gradua\c{c}\~ao mais baixa, cubra a panela e deixe cozinhar em fogo brando por 12-15 minutos. A cada 3 minutos remova a tampa e mexa com uma colher de pau.
	 \item Remova do fogo, a mistura deve estar grossa, adicione uma x\'{\i}cara de leite frio.
	 \item  Adicione o queijo parmes\~ao ralado, a pimenta do reino mo\'{\i}da na hora e duas colheres de manteiga e misture bem. 
	 \item Se n\~ao for servir imediatamente, deixe a polenta na panela coberta.
	 \item Se a polenta esfriou bastante, quando chegar perto da hora de servir, esquente de novo.
	 \item Se a polenta tiver engrossado significantemente, adicione mais uma x\'{\i}cara de leite para deixar a polenta com uma consist\^encia cremosa de novo. 
	 \end{itemize}
	\item {\bf Manteiga Tostada com S\'alvia}
	\begin{itemize}
        		\item Transfira a polenta cremosa para um prato untado com manteiga. 
		\item Polvilhe a superf\'{\i}cie com pimenta do reino mo\'{\i}da na hora. 
		\item Use a parte de traz de uma colher de sopa para deixar a superf\'icie bem lisa. Fa\c{c}a marcas em cruzado sobre a superf\'icie com a ponta de uma faca.
		\item Empilhe as folhas restantes de s\'alvia e corte em tirinhas muito finas. 
		\item Coloque as restantes quatro colheres de manteira em uma frigideira (\'e melhor usar uma frigideira de a\c{c}o inoxid\'avel ou uma frigideira de cor clara porque \'e dif\'icil ver a cor da manteiga em uma frigideira escura).
		\item Cozinhe a manteira em fogo moderadamente alto at\'e que ela adquira uma cor marrom clara. Voc\^e precisa observar cuidadosamente para permitir que a manteiga adquira um sabor de tostada sem queimar. Quando ela estiver no ponto, remova imediatamente do fogo e adicione a s\'alvia. 
		\item Espalhe a manteiga tostada e a s\'alvia sobre a superf\'{\i}cie da polenta cremosa.  
		\item Sirva morna ou em temperatura ambiente.
	\end{itemize}
     	\end{enumerate}         
\end{description}
\end{document}



