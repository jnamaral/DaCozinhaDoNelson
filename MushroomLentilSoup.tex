
\documentclass[11pt,letterpaper]{article}
\input{headings}
\newcommand \recipeName {Sopa de Lentilha com Cogumelos}
\chead{\recipeName}

\begin{document}
\input{title}

Criei a versão com bacon desta receita numa noite, simplesmente reunindo os ingredientes que tinha à mão, quando decidi convidar uma amiga para jantar durante a semana e não tinha tempo para ir ao supermercado. Ficou muito boa. Quando convidamos vários vizinhos para uma tarde de confraternização com sopa, pão e bebidas, adaptei a receita para ser a opção vegana. Listo as duas versões abaixo. Gosto de cozinhar as cenouras e os cogumelos separadamente e adicioná-los no último minuto para que mantenham seu sabor e cor característicos na sopa.

\vspace{0.3in}

\begin{description}

\item[Ingredients (bacon version):]\ \\
	\begin{itemize}
\item 1 xícara de lentilhas marrons
\item 2 colheres de sopa de óleo de cozinha
\item 2 colheres de sopa de manteiga
\item 225-280 g de cogumelos (1 pacote), sejam eles champignon ou cremini
\item 3-4 cenouras (para obter cerca de 1 1/2 xícara após picadas)
\item 1/2 colher de chá de açúcar
\item sal
\item pimenta-do-reino
\item 20 g de cogumelos secos (porcini ou outros)
\item 6-8 fatias de bacon
\item 3/4 de xícara de cebola picada
\item 1/2 xícara de aipo picado
\item 2 dentes de alho amassados ​​até formar uma pasta
\item 1 colher de chá de estragão seco
\item salsa
\item 2-3 xícaras de caldo de galinha (dependendo da quantidade de água restante nas lentilhas após o molho)
\item agente espessante (arroz cozido, farinha ou amido de milho)
\item Harissa ou outro molho picante
\end{itemize}

\item[Ingredientes (versão com bacon):]\ \\
	\begin{itemize}
	\item omita o bacon
	\item substitua o caldo de galinha por caldo de legumes
	\item use gordura vegetal para refogar os cogumelos e os legumes
	\end{itemize}

\item[Procedimento:]\ \\
\begin{enumerate}
	\item {\bf Preparo Inicial}
		\begin{itemize}
		\item Lave as lentilhas em uma peneira sobre água corrente fria.
		\item Coloque as lentilhas em uma tigela e cubra com água, de forma que haja cerca de 1,2 cm de água acima das lentilhas. Deixe de molho por pelo menos uma hora, ou até 12 horas.
		\item Descasque as cenouras e corte-as em cubos — o ideal é que os cubos tenham aproximadamente o tamanho de uma ervilha grande.
		\item Coloque as cenouras em cubos em uma tigela, polvilhe com 1/2 colher de chá de sal e 1/2 colher de chá de açúcar. Deixe descansar por 30 minutos a 1 hora.
		\item Cubra os cogumelos secos com água e deixe de molho por pelo menos 10 minutos.
		\item Corte os cogumelos em cubos de 1,25 cm.
		\item Corte a cebola e o aipo em cubinhos bem pequenos.
		\item Corte as fatias de bacon em pedaços de 2,5 cm.
		\end{itemize}

	\item{\bf Cozinhar os cogumelos}
		\begin{itemize}
		\item Em uma panela grande, aqueça uma colher de sopa de óleo vegetal e uma colher de sopa de manteiga.
		\item Assim que estiver quente, adicione os cogumelos em cubos, espalhe-os e deixe dourar de um lado primeiro — o objetivo é obter uma cor dourada clara.
		\item Mexa os cogumelos até que estejam cozidos, mas não ressecados. Quando os cogumelos estiverem cozidos, polvilhe-os com um pouco de sal e pimenta-do-reino.
		\item Coloque o estragão em uma das mãos e esmague-o com a outra sobre os cogumelos, mexendo.
		\item Transfira os cogumelos para uma tigela e reserve.
		\end{itemize}
 	\item {\bf Cenouras Salteadas}
		\begin{itemize}
		\item Em uma panela grande, aqueça uma colher de sopa de óleo vegetal e uma colher de sopa de manteiga.
		\item Assim que estiver quente, adicione as cenouras em cubos — retire-as da tigela com uma escumadeira, deixando a água na tigela.
		\item Espalhe as cenouras em cubos na panela e deixe cozinhar do lado de baixo até que mudem de cor.
		\item Comece a mexer as cenouras e continue cozinhando até que reduzam de volume e adquiram uma leve cor dourada.
		\item Prove um pedaço de cenoura para verificar o sal. Se precisar de mais sal, adicione um pouco da água que as cenouras soltaram na tigela.
		\item Retire as cenouras cozidas para uma tigela e reserve.
		\end{itemize}
	\item {\bf Lentilhas Cozidas - versão com bacon}
		\begin{itemize}
		\item Coloque o bacon em uma frigideira e cozinhe por alguns minutos — não deixe ficar crocante, ele deve permanecer macio.
		\item Adicione a cebola e o aipo e cozinhe por alguns minutos até que os vegetais estejam macios.
		\item Com uma faca, amasse o alho até formar uma pasta e adicione aos vegetais. Refogue por 30 segundos.
		\item Adicione as lentilhas previamente hidratadas (elas devem ter absorvido a maior parte da água).
		\item Adicione o caldo.
		\item Pique os cogumelos secos reidratados e adicione às lentilhas. Adicione o líquido aos poucos para evitar que sedimentos arenosos se acumulem no fundo da panela e sejam adicionados à sopa.
		\item Cozinhe em fogo baixo por cerca de 30 minutos, até que as lentilhas estejam macias.
		\end{itemize}
	\item {\bf Lentilhas Cozidas - versão vegana}
		\begin{itemize}
		 \item Em vez de bacon, use duas colheres de sopa de óleo de cozinha. Proceda como acima.
		\end{itemize}
\item {\bf Engrossando a sopa}
		\begin{itemize}
		\item Existem algumas opções para engrossar a sopa. Se o glúten não for um problema, misture farinha com manteiga em temperatura ambiente e adicione à sopa (uma colher de sopa de cada deve ser suficiente).
		\item Para uma versão sem glúten, a melhor opção é processar arroz branco cozido com o líquido da sopa em um liquidificador até ficar bem homogêneo e adicioná-lo de volta à sopa — foi o que eu fiz quando preparei uma versão vegana sem glúten para um grupo grande.
		\item Como alternativa, misture uma colher de sopa de amido de milho com duas colheres de sopa de água e adicione à sopa.
		\item Você também pode amassar algumas lentilhas com um garfo e adicioná-las de volta à sopa.
		\end{itemize}
\item {\bf Finalizando e servindo}
		\begin{itemize}
		\item Adicione as cenouras cozidas e os cogumelos refogados à sopa.
		\item Adicione a salsinha picada.
		\item Adicione harissa ou seu molho picante favorito a gosto.
		\item Prove e ajuste o sal e a pimenta.
		\item Cozinhe em fogo baixo por cerca de 30 minutos, até que as lentilhas estejam macias.
		\end{itemize}
	\end{enumerate}
\end{description}
\end{document}



