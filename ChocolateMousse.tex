\documentclass [11pt, letterpaper] {article}
\input {headings}
\usepackage{soul}
\newcommand \recipeName {Mousse de Chocolate}
\chead {\recipeName}

\begin {document}
\input {title}

 Esta \'e uma vers\~ao de mousse de chocolate sem leite. A mesma sobremesa pode ser preparada com leite e manteiga no lugar de leite de am\^endoa e margarina. .
A base para este mousse \'e um creme que eu aprendi a fazer com a Bebe Brown, a av\'o do Scott, na cozinha dela em Oklahoma nos anos 1990s. Ela sempre fez tortas de cremes perfeitas. Durante um dos feriados de Natal eu fiquei atras dela enquanto ela preparava a torta sem olhar a receita, e eu anotei o que ela fazia. 

\vspace{0.3in}
\begin{description}

\item[Ingredientes:]\ \\
	\begin{itemize}
	\item 3 x\'{\i}caras de leite de amend\^oa  (722 gramas)
	\item 1 x\'{\i}cara de a\c{c}ucar (200 gramas)
	\item 2 colheres de sopa de amido de milho (15 grams)
	\item 1/4 x\'{\i}cara de farinha (35 grams)
	\item 1/4 colherinha de ch\'a de sal
	\item 4 ovos
	\item 3 colheres de sopa de margarina
	\item 1/2 colherinha de ch\'a de baunilha 
	\item \st{1/2 x\'{\i}cara de polpa de maracuj\'a}
	\item 125 gramas de chocolate amargo de boa qualidade (70\% de cacao)
	\item 1 colher de sopa de felatina em p\'o sem sabor
	\item 1/4 colherinha de ch\'a de creme de tartar
	\end{itemize}
	
	{\bf Varia\c{c}\~ao com Caf\'e e Rum}
	\begin{itemize}
	\item um caf\'ezinho expresso
	\item 1 colher de sopa de rum.
	\end{itemize}
	
\item[Procedimento:]\ \\
	\begin{enumerate}
	\item {\bf Misture a base para o creme}
		\begin{itemize}
		\item Em uma panela de fundo grosso misture com um bal\~ao a farinha, o amido de milho, o sal, e 1/2 x\'{\i}cara de a\c{c}ucar 
		\item Me\c{c}a o leite em uma vasilha que possa ir no microondas.
		\item Adicione gotas do leite de am\^endoas frio na mistura de farinha mexendo bem com o bal\~ao at\'e que voc\^e tenha uma pasta fina e sem bolotas --- o segredo \'e adicionar uma pequena quantidade de leite de cada vez.
		\item Aque\c{c}a o restante do leite de am\^endoas no microondas at\'e que esteja bem quente.
		\end{itemize}

	\item {\bf Prepare a Base de Gelatina (1)}
		\begin{itemize}
		\item Em uma pequena vasilha polvilhe a gelatina sobre 1/3 de x\'{\i}cara de leite de am\^endoa e deixe parada.
		\end{itemize}
	\item {\bf Prepare a Base de Gelatina (2) --- varia\c{c}\~ao com caf\'e e rum}
		\begin{itemize}
		\item Prepare um caf\'ezinho expresso, coloque em uma pequena vasilha e deixe esfriar at\'e a temperatura ambiente.
		\item Polvilhe a gelatina sobre o espresso frio e deixe parado.
		\item Depois que a gelatina estiver molhada, adicione uma colher de sopa de rum.
		\end{itemize}

	\item {\bf Derreta o chocolate}
		\begin{itemize}
		\item Quebre o chocolate em pequenos peda\c{c}os em uma pequena vasilha que possa ir no microondas. A vasilha deve estar bem limpa e bem seca --- uma gota de \'agua pode fazer of chocolate endurecer e estraga o chocolate.
		\item Derreta o chocolate no microondas cozinhando em alta pot\^encia um minuto de cada vez --- usualmente leva dois minutos, mas depende do microondas, pode levar at\'e cinco minutos. Pare quando ainda tiverem alguns pequenos peda\c{c}os de chocolate que n\~ao est\~ao completamente derretidos. Misture com uma pequena colher limpa e deixe parado. O calor residual vai terminar de derreter o chocolate. Se o chocolate ficar bem quente, deixe esfriar um pouquinho antes de prosseguir.
		\end{itemize}
		
	\item {\bf Desmanche as Gemas}
		\begin{itemize}
		\item Separe os ovos colocando as gemas em uma pequena vasilha e as claras no prato de uma batedeira. 
		\item Fure as gemas com um garfo e misture. 
		\item Quando o leite de am\^endoas estiver quente, adicione gotas do leite quente nas gemas mexendo constantemente para as gemas n\~ao embolarem. 
		\item Adicione leite de am\^endoas suficiente para que as gemas fiquem fluidas --- em torno de 1/3 de x\'{\i}cara.
		\end{itemize}

	\item {\bf Cozinhe o creme}
		\begin{itemize}
		\item Mexendo vigorosamente com o bal\~ao, despeje um fio do leite de am\^endoas quente sobre a mistura de farinha at\'e ela ficar bem l\'iquida.
		\item Despeje o  restante do leite quente e misture.  
		\item Cozinhe a mistura em um fogo m\'edio mexendo constantemente inicialmente com o bal\~ao e depois com uma colher de pau. 
		\item Continue cozinhando at\'e que a mistura esteja borbulhando levemente e tenha engrossado. 
		\item Desligue o fogo. 
		\item Mexendo constantemente, adicione a mistura de gemas e leite em um fio. 
		\item Adicione a mistura de leite de am\^endoa  e  gelatina ao creme. 
		\end{itemize}
	
	\item {\bf Coe o creme}
		\begin{itemize}
		\item Coloque um coador grande sobre uma vasilha funda. 
		\item Despeje o creme no coador e use uma esp\'atula ou uma colher para fazer o creme passar pelo coador.
		\end{itemize}
	
	\item {\bf Termine o Creme}
		\begin{itemize}
		\item Incorpore o chocolate derretido no creme depois de coar e antes de adicionar a margarina.
		\item Adicione a margarina e mexa at\'e que ela derreta. 
		\item Adicione a baunilha no creme e misture. 
		\item Deixe o creme esfriar, mexendo de vez em quando, at\'e que o creme esteja morno.
		\end{itemize}

	\item {\bf Bata e incorpore as claras (1)  --- metodo simples }
		\begin{itemize}
		\item Adicione o creme de tartar nas claras.
		\item Bata as claras na batedeira at\'e formar picos suaves.
		\item Continue batendo e lentamente adicione a 1/2 x\'{\i}cara de a\c{c}ucar restante.
		\end{itemize}

	\item {\bf Bata e incorpore as claras (1)  --- varia\c{c}\~ao para um mousse mais est\'avel}
		\begin{itemize}
		\item Misture as claras com o a\c{c}ucar em uma vasilha que possa ir ao fogo --- o ideal \'e o prato met\'alico de uma batedeira el\'etrica.
		\item Coloque o prato diretamente no fogo ou sobre uma panela de \'agua fervente.
		\item Bata as claras e o a\c{c}ucar com um bal\~ao sobre o fogo at\'e que a mistura esteja bem quente. Se voc\^e estiver usando o mesmo bal\~ao que usou para o creme, lave bem antes de usar nas claras. Mesmo uma pequena quantia de gordura ou impureza pode resultar em as claras n\~ao ganharem volume.
		\item Transfira para  a batedeira e bata o merengue at\'e se formar picos suaves.
		\end{itemize}
	
	\item {\bf Misture o merengue no creme}
		\begin{itemize}
		\item Primeiro adicione uma x\'{\i}cara do merengue no creme e misture com uma esp\'atula para amolecer o creme. 
		\item Adicione o restante do merengue e misture com o creme usando uma esp\'atula e fazendo uma seq\"u\^encia de movimentos onde voc\^e levanta o creme do fundo do prato e roda o prato at\'e que o merengue esteja completamente incorporado no creme.
		\item Coloque em pratinhos pequenos individuais ou em um prato grande.
		\end{itemize}
	
	\item {\bf Refrigere}
		\begin{itemize}
		\item Coloque no refrigerador, sem cobrir, por 30 a 45 minutos para esfriar completamente.
		\item Coloque um filme de pl\'astico, sem tocar o mousse, sobre cada pratinho ou sobre o prato grande, para evitar que o mousse fique requecido em cima.
		\end{itemize}
	\end{enumerate}

\end{description}
\end{document}
