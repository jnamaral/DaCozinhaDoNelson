\documentclass [11pt, letterpaper] {article}
\input {headings}
\newcommand \recipeName {Sopa de Lentilha Vermelha}
\chead {\recipeName}

\begin {document}
\input {title}

\begin {flushright}
Da {\it America's Test Kitchen}
\end {flushright}

\begin {description}

\item [Ingredientes:] \ \\
\begin {itemize}
\item 4 colheres de sopa de manteiga sem sal
\item 1 cebola grande, cortada bem
\item Sal e pimenta
\item 3/4 colher de chá de coentro moído
\item 1/2 colher de chá de cominho moído
\item 1/4 colher de chá de gengibre moído
\item 1/8 colher de chá de canela molhada
\item Pinch cayenne
\item 1 colher de sopa de pasta de tomate
\item 1 dente de alho, triturado
\item 4 xícaras de caldo de galinha
\item 2 xícaras de água
\item 10 1/2 onças (1 1/2 xícaras) de lentilhas vermelhas, colhidas e enxaguadas
\item 2 colheres de sopa de suco de limão, mais extra para tempero
\item 1 1/2 colheres de chá de menta seca, amassada
\item 1 colher de chá de paprika
\item 1/4 xícara de coentro fresco cortado
\end {itemize}

\item [Procedimento:] \ \\
\begin {description}
\item [Preparar a Base] \ \\
\begin {itemize}
\item Derreta 2 colheres de sopa de manteiga em uma panela grande a fogo médio.
\item Adicione a cebola e 1 colher de chá de sal e cozinhe, mexendo ocasionalmente, até suavizar, mas não dourado, cerca de 5 minutos.
\item Adicione coentro, cominho, gengibre, canela, caiena e 1/4 de colher de chá de pimenta e cozinhe até perfumar, cerca de 2 minutos.
\item Misture a massa de tomate e alho e cozinhe por 1 minuto.
\item Mexa em caldo, água e lentilhas e leve a ferver. Misture com força vigorosamente, mexendo de vez em quando, até que as lentilhas sejam macias e cerca de metade são quebradas, cerca de 15 minutos.
\end {itemize}
\item [Creme a sopa] \ \\
\begin {itemize}
\item Pule sopa vigorosamente até que seja grosseiramente purificada, cerca de 30 segundos.
\item Misture o suco de limão e tempere com sal e suco de limão extra para saborear.
\item Cubra e mantenha quente. (A sopa pode ser refrigerada por até 3 dias. Sopa fina com água, se desejado, ao reaquecer.)
\end {itemize}
\item [Decorar a sopa] \ \\
\begin {itemize}
\item Derreta restante 2 colheres de sopa de manteiga em uma frigideira pequena.
\item Retire do fogo e mexa em menta e paprika.
\item Sopa de sopa em tigelas individuais, drizzle cada porção com 1 colher de chá de manteiga temperada, polvilhe com coentro e sirva.
\end {itemize}
\end {description}
\end {description}
\end {document}
