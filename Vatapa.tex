\documentclass [11pt, letterpaper] {article}
\input {headings}
\newcommand \recipeName {Vatap\'a}
\chead {\recipeName}

\begin {document}
\input {title}

Vatap\'a é um prato da Bahia no Nordeste do Brasil. As raízes são africanas. A base para o prato é o pão embebido em leite, com uma combinação muito interessante de ingredientes típicos da região do Brasil. Vatap\'a é um recheio tradicional para Acaraj\'e. Em Edmonton, você pode encontrar mandioca ralada crua congelada na SuperStore.

\vspace {0.3in}

\begin {description}

\item [Ingredientes:] \ \\
\begin {itemize}
\item leite
\item 1 baguette francês ou pão italiano
\item pequena quantidade de camarão seco
\item 3 colheres de sopa de manteiga de amendoim não açucarada
\item 250 gramas de mandioca ralada
\item 1 cebola grande
\item 3 dentes de alho
\item 5 cent\'imetros de gengibre
\item 8 tomates inteiros de uma lata
\item 1 lata de leite de coco
\item salsa fresca
\item 1/2 xícara de dend\^e óleo
\item 500 gramas de camarão fresco
\end {itemize}

\item [Procedimento:] \ \\

\begin {enumerate}
\item {\bf Preparando a base}
\begin {itemize}
\item Corte o pão em fatias, coloque em uma tigela grande e despeje bastante leite para absorver o pão completamente
\item Descasque e corte a cebola em fatias grossas.
\item Descasque e corte o gengibre em fatias.
\item Descasque os dentes de alho.
\item Adicione as cebolas, o gengibre e o alho a um processador de alimentos de bom tamanho e processe-o até obter um l\'iquido.
\item Adicione o pão embebido, a mandioca ralada, a manteiga de amendoim e os tomates ao processador de alimentos.
\item Processe, raspando os lados da tigela de vez em quando, até que você tenha uma pasta homogênea.
\end {itemize}

\item {\bf Cozinhe a base}
\begin {itemize}
\item Transfira a pasta para uma panela de fundo grosso e cozinhe em fogo moderado, mexendo de vez em quando para evitar que grude ao fundo da panela e queime, até que a pasta fique grossa e comece a libertar-se do fundo da panela.
\end {itemize}

\item {\bf Concluir o vatap\'a}
\begin {itemize}
\item Acrescente o camarão seco, o leite de coco, o óleo de dend\^e e deixe o vatap\'a ficar muito quente novamente, mas sem ferver
\item Desligue o calor, adicione o camarão fresco e a salsa picada e cubra o pote por 3-5 minutos até o camarão esteja cozido no calor residual.
\end {itemize}

\end {enumerate}
\end {description}
\end {document}

