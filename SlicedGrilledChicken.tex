\documentclass [11pt, letterpaper] {article}
\input {headings}
\newcommand \recipeName {Frango Grelhado em Fatias}
\chead {\recipeName}

\begin {document}
\input {title}

\begin {description}

\item [Ingredientes:] \ \\
\begin {itemize}
\item Peitos de frango sem pele desossados
\item Limão
\item Sal
\item Açúcar
\item Pimenta recém-molhada
\item Óleo de cozinha sem sabor, como óleo de canola
\item Óleo de oliva virgem extra
\item Tomilho fresco (ou salsa)
\end {itemize}

\item [Procedimento:] \ \\
\begin {enumerate}
\item {\bf Cortar os peitos de frango para torná-los mais parelhos para a grelha (pelo menos três horas antes de grelhar e até três dias antes).}
\begin {itemize}
\item Remova os lombinhos dos peitos.
\item Corte uma pequena forma triangular do final de cada peito para remover a parte fina do peito.
\item Coloque cada peito do seu lado e corte uma fatia fina da parte mais grossa para tornar cada peito um pouco mais uniforme.
\item Lave o peito de frango em água corrente fria.
\end {itemize}

\item {\bf Brining the chicken}
\begin {itemize}
\item Faça uma solução de salmoura com as seguintes proporções: para cada quarto de água, adicione 1/4 xícara de sal de mesa e 1/4 xícara de açúcar.
\item Adicione uma quantidade generosa de pimenta preta recém-mo\'ida à salmoura.
\item Adicione raspas de um ou dois limões à salmoura (melhor usar um raspador de limão, mas você também pode usar um pequeno ralo). Se você ralar o lim\~ao em cima da água você vai capturar alguns dos óleos essenciais de limão na salmoura.
\item Adicione todo o frango à salmoura e deixe descan\c{c}ar, na geladeira, por pelo menos duas horas, mas você também pode mantê-los na salmoura durante a noite.
\item Retire o frango da salmoura e coloque em um escorregador de massa ou em uma grade dentro de uma pia limpa para deixar toda a salmoura escorrer.
\item Se você estiver fazendo a salmoura um dia ou dois à frente, coloque o frango escorrido na geladeira e deixe-o descoberto por várias horas para ajudá-lo a continuar a secar. Após 10 ou 12 horas, cubra o frango para que não fique muito seco na geladeira.
\end {itemize}
\item {\bf Finalizando o marinado}
\begin {itemize}
\item Cerca de uma hora antes de planejar começar a grelhar, adicione um pouco de limão de limão fresco, pimenta fresca e o suco de um limão para o frango.
\end {itemize}

\item {\bf Grelhando o frango}
\begin {itemize}
\item Obtenha sua grade tão quente quanto possível (minha grade de gás chega a 550F)
\item Escove a grelha com uma escova de grade rígida para se certificar de que está limpo
\item Despeje uma pequena quantidade de óleo em um prato pequeno, dobre um pedaço de toalha de papel várias vezes e segure a toalha de papel dobrada com as línguas da cozinha, profundamente no óleo e esfregaço por toda a grade. Cubra a grelha para deixar o óleo arder por um minuto. Repita o processo três ou quatro vezes para reduzir a viscosidade da grelha.
\item Grelhe o frango, deixando o primeiro lado mais longo do que o segundo lado.
\item À medida que cada pedaço de bife é feito (a temperatura interna deve estar entre 140 F e 145 F em um termômetro de leitura instantânea), remova-o para um prato coberto (uma tigela coberta com uma placa de jantar ou uma panela pesada com chumbo). É muito importante não cozinhar demais o frango, pois continuará a cozinhar.
\item Mantenha o frango cozido coberto até esfriar o suficiente para manusear (20 a 30 minutos). O frango irá liberar uma quantidade significativa de suco enquanto repousa. Certifique-se de preservar o suco.
\end {itemize}

\item {\bf Cortando o frango para servir}
\begin {itemize}
        \item Tente selecionar uma taboa de corte na qual seja fácil coletar os sucos do frango. Ao cortar o frango, coloque os sucos de volta no recipiente onde est\'a a galinha.
\item Usando uma faca grande afiada ou uma faca de pão afiada, corte o frango muito finamente e coloque de volta na panela com os sucos enquanto você corta-os.
\item Adicione algumas colheres de sopa de azeite virgem extra ao frango.
\item Adicione folhas de tomilho frescas  \`a galinha e misture bem (melhor usar as mãos se fizer uma quantidade grande de frango) --- se você não tem tomilho fresco, você pode usar salsa italiana fresca.
\item Prove para decidir se você acha que precisa de mais suco de limão ou mais azeite  de oliva virgem ou mais pimenta moída.
\item Sirva à temperatura ambiente ou ainda um pouco quente.
\end {itemize}
     \end {enumerate}
\end {description}
\end {document}
