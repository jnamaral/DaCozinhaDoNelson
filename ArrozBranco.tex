\documentclass [11pt, letterpaper] {article}
\input {headings}
\newcommand \recipeName {Arroz Branco}
\chead {\recipeName}

\begin {document}
\input {title}

O arroz branco é um prato diário no Brasil. Em muitas partes do Brasil, o alho é adicionado ao arroz branco. Mas não no Sul. Assim, seguiremos a tradição do Sul e deixamos o alho de fora. A receita aqui é para uma xícara de arroz que serve três pessoas. Ao expandir a receita, dimensione a quantidade de água e arroz linearmente, mas não balanceie o sal linearmente. Você também terá que experimentar o tempo de cozimento para maiores quantidades. Meu arroz favorite \'e o arroz tipo Jasmin da Tail\^andia.

\vspace {0.3in}

\begin {description}

\item [Ingredientes:] \ \\
\begin {itemize}
\item 1 xícara de arroz
\item 2 xícaras de água (480 ml)
\item 1 1/2 colheres de sopa de óleo de canola
\item 1 colher de sopa de sal
\end {itemize}

\item [Procedimento:] \ \\

\begin {enumerate}
\item {\bf Sautee e ferver o arroz}
\begin {itemize}
\item Aqueça a água no microondas até que ferva --- quatro a cinco minutos.
\item Enquanto isso, em uma panela pequena, coloque o óleo de cozinha e o arroz e cozinhe com fogo alto, mexendo constantemente até que os grãos de arroz comecem a ficar Opacos.
\item Reduza o fogo para o fogo lento mais lento.
\item Despeje a água quente sobre o arroz mexendo.
\item Cubra a panela e cozinhe por 20 minutos. Não abra o pote.
\item Após 20 minutos, desligue o fogo e mantenha o recipiente fechado por mais cinco minutos.
\item Não mexa o arroz. Você pode afrouxar levemente com um garfo.
\end {itemize}

\end {enumerate}
\end {description}
\end {document}
