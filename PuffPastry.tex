\documentclass [11pt, letterpaper] {article}
\input {headings}
\newcommand \recipeName {Massa Folhada}
\chead {\recipeName}

\begin {document}
\input {title}

\begin {flushright}
De {\it The French Chef} por Julia Child.
\end {flushright}
 
Esta é a melhor receita para Massa Folhada que eu já fiz. Não é difícil, mas você precisa ler a receita com cuidado antes de começar. Os tempos de descanso da massa são muito importantes porque uma massa folhada de sucesso é alcançada quando os componentes e a massa são mantidos resfriados em todos os momentos.

\begin {description}

\item [Ingredientes:] \ \\
\begin {itemize}
\item 2 3/4 xícaras de farinha (380 gramas)
\item 3/4 xícaras de farinha de bolo (85 gramas)
\item 2 colheres de chá de sal
\item 1/4 xícara de óleo de cozinha sem sabor (50 gramas)
\item 1 xícara de água gelada (250 gramas)
\item 340 gramas de manteiga sem sal em tabletes refrigerada
\end {itemize}

\item [Procedimento:] \ \\
\begin {enumerate}
\item {\bf Misture a massa}
\begin {itemize}
\item Em uma tigela, misture as farinhas com uma espátula de borracha.
\item Remova 1/2 xícara de farinhas misturadas e reserve para mais tarde.
\item Distribua o óleo sobre a farinha em um fio e misture bem com uma espátula de borracha.
\item Dissolva o sal na água gelada.
\item Adicione a água de uma só vez.
\item Cortando e mexendo com a espátula, misture até que a massa se junte.
\item Use os dedos para juntar a massa.
\item Transfira a massa para a superfície de trabalho e pressione tudo em uma bola.
\item Enrole a massa em papel encerado.
\item Coloque em um saco plástico.
\item Refrigere   por pelo menos 40 minutos.
\end {itemize}


\item {\bf Prepare a manteiga}
\begin {itemize}
\item Coloque os tabletes de manteiga lado a lado no balcão.
\item Bata a manteiga com um rolo até que tenha cerca de 1 cent\'imetro de espessura.
\item Polvilhe a 1/2 xícara de farinha misturada que voce reservou em cima da manteiga.
\item Usando apenas a parte the tras da m\~ao --- não use a palma da mão para n\~ao derreter a manteiga demais --- pressione a manteiga na superfície de trabalho para incorporar a farinha na manteiga.
\item Usando o scrapper do banco, a manteiga em um retângulo de 38 x 30 cent\'imetros.
\item Abra um pedaço de papel encerado no balcão.
\item Usando o raspador de massa,\footnote{Utens\'ilio de cozinha chamado "bench scraper" em ingl\^es, que \'e essencial para trabalhar com massas. Se n\~ao tiver pode usar uma faca grande de cozinha} levante o retângulo da manteiga para o papel encerado.
\item Coloque na geladeira por pelo menos 20 minutos.
\end {itemize}
\item {\bf Adicione a manteiga à massa}
\begin {itemize}
\item Enfarinhe levemente sua superfície de trabalho.
\item Abra a massa resfriada pressionando com as mãos.
\item Usando o rolo estique em um retângulo de 40 x 45 cent\'imetros com o lado de 40 cent\'imetros na sua frente. 
\item Coloque o retângulo de manteiga na extremidade do retângulo da massa mais distante de você, a manteiga deve cobrir 2/3 do retângulo da massa. Deixe 1 cent\'imetro  ao redor do retângulo sem cobrir com a manteiga. O 1/3 do retângulo de massa que não está coberto pela manteiga está mais perto de você e é chamado de aba.
\item Dobre a aba sobre a manteiga para que metade da manteiga seja coberta pela aba.
\item Dobre o retângulo resultante pela metade para que agora toda a manteiga esteja coberta pela massa.
\item Pressione suavemente com as pontas dos dedos ao redor para selar a manteiga dentro da massa.
\item Gire a massa um quarto de volta para que o lado mais curto esteja na sua frente.
\end {itemize}
\item {\bf Faça a dobra de quatro camadas}
\begin {itemize}
\item Pressione com o rolo em intervalos iguais à largura do seu rolo começando na extremidade perto de você fazendo pequenos valos ao longo de toda a massa até a outra extremidade.
\item Usando o rolo estique a massa para um retângulo de 16 x 8 polegadas.
\item Se houver buracos na massa que expõem a manteiga, polvilhe-os com um pouco de farinha.
\item Dobre as duas extremidades longas do retângulo para o centro para que as duas arestas se encontrem no centro.
\item Dobre novamente pela metade para que essas bordas não sejam mais visíveis.
\item Pressione com dois dedos no meio do retângulo dobrado para que você lembrar que completou a segunda dobra.
\item Voc\^e completou 12 camadas de manteiga.
\item Enrole a massa no papel encerado, escorra uma bolsa de plástico e coloque a geladeira por pelo menos 40 minutos.
\end {itemize}


\item {\bf Faça as dobras 3 e 4}
\begin {itemize}
\item Remova a massa gelada do pl\'astico e polvilhe levemente com farinha em ambos os lados.
\item Faça indentações com seu rolo na massa. Primeiro longitudinalmente e depois transversalmente. Isso tornará mais fácil rolar a massa gelada.
\item Usando o rolo estique a massa para criar um retângulo de 40 x 45 cent\'imetros.
\item Dobre este retângulo em três exatamente como você dobra uma carta de negócios.
\item Rode a massa um quarto de volta, de modo que o lado mais curto do retângulo dobrado esteja na sua frente.
\item Estique a massa novamente para um retângulo de 40 x 45 cent\'imetros.
\item Dobre novamente como uma carta comercial.
\item Faça quatro indentações no topo.
\item Enrole no papel de cera, coloque no saco plástico e coloque na geladeira.
\item Esta massa tem 72 camadas de manteiga e está pronta para usar. É necessário refrigerar por duas horas antes de usar. Ele durará na geladeira por até cinco dias e pode ser congelada.
\end {itemize}
\item {\bf Faça as dobras 5 e 6 --- para uma massa mais folhada}
\begin {itemize}
\item Após a massa refrigerada durante pelo menos 40 minutos, repita as voltas 3 e 4 para criar 648 camadas de manteiga.
\item Refrigere por duas horas antes de usar a massa.
\end {itemize}
\end {enumerate}
\end {description}

\begin{table}
\begin{tabular}{cccc}
\includegraphics[width=0.25\textwidth]{\imageDir/PuffPastry/IMG_8466.jpg} &
\includegraphics[width=0.25\textwidth]{\imageDir/PuffPastry/IMG_8468.jpg} &
\includegraphics[width=0.25\textwidth]{\imageDir/PuffPastry/IMG_8470.jpg} &
\includegraphics[width=0.25\textwidth]{\imageDir/PuffPastry/IMG_8473.jpg} \\
\includegraphics[width=0.25\textwidth]{\imageDir/PuffPastry/IMG_8476.jpg} &
\includegraphics[width=0.25\textwidth]{\imageDir/PuffPastry/IMG_8477.jpg} &
\includegraphics[width=0.25\textwidth]{\imageDir/PuffPastry/IMG_8478.jpg} &
\includegraphics[width=0.25\textwidth]{\imageDir/PuffPastry/IMG_8483.jpg} \\
\includegraphics[width=0.25\textwidth]{\imageDir/PuffPastry/IMG_8485.jpg} &
\includegraphics[width=0.25\textwidth]{\imageDir/PuffPastry/IMG_8486.jpg} &
\includegraphics[width=0.25\textwidth]{\imageDir/PuffPastry/IMG_8487.jpg} &
\includegraphics[width=0.25\textwidth]{\imageDir/PuffPastry/IMG_8488.jpg} \\
\includegraphics[width=0.25\textwidth]{\imageDir/PuffPastry/IMG_8489.jpg} &
\includegraphics[width=0.25\textwidth]{\imageDir/PuffPastry/IMG_8490.jpg} &
\includegraphics[width=0.25\textwidth]{\imageDir/PuffPastry/IMG_8491.jpg} &
\includegraphics[width=0.25\textwidth]{\imageDir/PuffPastry/IMG_8492.jpg} \\
\end{tabular}
\end{table}

\begin{table}
\begin{tabular}{cccc}
\includegraphics[width=0.25\textwidth]{\imageDir/PuffPastry/IMG_8493.jpg} &
\includegraphics[width=0.25\textwidth]{\imageDir/PuffPastry/IMG_8494.jpg} &
\includegraphics[width=0.25\textwidth]{\imageDir/PuffPastry/IMG_8496.jpg} &
\includegraphics[width=0.25\textwidth]{\imageDir/PuffPastry/IMG_8497.jpg} \\
\includegraphics[width=0.25\textwidth]{\imageDir/PuffPastry/IMG_8498.jpg} &
\includegraphics[width=0.25\textwidth]{\imageDir/PuffPastry/IMG_8499.jpg} &
\includegraphics[width=0.25\textwidth]{\imageDir/PuffPastry/IMG_8515.jpg} &
\includegraphics[width=0.25\textwidth]{\imageDir/PuffPastry/IMG_8516.jpg} \\
\includegraphics[width=0.25\textwidth]{\imageDir/PuffPastry/IMG_8517.jpg} &
\includegraphics[width=0.25\textwidth]{\imageDir/PuffPastry/IMG_8518.jpg} &
\includegraphics[width=0.25\textwidth]{\imageDir/PuffPastry/IMG_8519.jpg} \\
\end{tabular}
\end{table}

\end {document}
