\documentclass [11pt, letterpaper] {article}
\input {headings}
\newcommand \recipeName {Bolinho Mandioca}
\chead {\recipeName}

\begin {document}
\input {title}

\begin{flushright}
(Do livro {\it Dona Benta: Comer Bem})
\end{flushright}

\begin {description}

\item [Ingredientes:] \ \\
\begin {itemize}
\item Mandioca
\item Óleo de canola
\item 2 ovos
\item 1 colher de chá de fermento em pó
\item Sal
\item Pimenta preta recém-moída
\item Água
\end {itemize}

\item [Ingredientes (opcional para aromatizar, use uma seleção):] \ \\
\begin {itemize}
\item Bacon
\item Cebola
\item Queijo Parmesão
\item Ervas frescas (cebolinha, salsa, manjericão, alecrim, sálvia, tomilho)
\end {itemize}

\item [Procedimento:] \ \\
\begin {enumerate}
\item {\bf Descasque a mandioca}
\begin {itemize}
\item Encha sua pia com água quente da torneira e remova a mandioca em água morna por alguns minutos para afrouxar a pele.
\item Corte as extremidades de cada raiz de mandioca com uma faca grande.
\item Com uma faca de ponta pequena, descasque a mandioca começando em cima.
\item Corte a mandioca em segmentos de 3 a 4 polegadas de comprimento.
\item Suporte cada segmento e corte ao meio longitudinalmente.
\item Corte cada meio pela metade novamente.
\item Remova a cadeia central de cada segmento.
\item Remova todos os peda\c{c}os que estejam escuros, fibrosos ou duros.
\item Neste ponto, você pode submergir a mandioca descascada em água para que todos os pedaços de mandioca sejam cobertos por água, e coloque no congelador, ou você pode prosseguir a cozinhar.
\end {itemize}
\item {\bf Cozinhe a mandioca}
\begin {itemize}
\item Se você congelou a mandioca, coloque a mandioca congelada com a água circundante em uma panela com água  e coloque a fervura.
\item Caso contrário, deixe cair a mandioca pelada na água e deixe ferver. Não adicione nenhum tempero ou sal à água.
\item Ferva por 40-60 minutos até o mandioca ficar macio. Se ficar bem mole, você conseguiu uma boa mandioca.
\item Alternativamente, especialmente a partir de mandioca congelada, coloque a mandioca em uma panela de pressão, e cozinhe sob pressão por 15 minutos. Coloque a panela de pressão sob água fria e remova cuidadosamente a válvula de pressão para liberar o vapor.
\item Desligue o fogo, adicione sal à água e deixe reposar por 5 a 10 minutos.
\item Escorra a mandioca cozida em um coador de massa.
\end {itemize}
\item {\bf Mashing the mandioca}
\begin {itemize}
\item Deixe a mandioca esfriar completamente (você pode cozinhar no dia anterior e mantê-la na geladeira).
\item Remova o fiapo do centro dos pedaços de mandioca, bem como quaisquer peda\c{c}os duros que possam estar no final das peças cortadas depois de serem cozidas.
\item Em um prato grande, usando um garfo, amassd a mandioca cozida. Você precisa de duas xícaras de mandioca amassada.
\end {itemize}
\item {\bf Preparando a massa}
\begin {itemize}
\item Adicione dois ovos, fermento em pó, pimenta preta e aromas (veja abaixo) para a massa e misture bem.
\end {itemize}
\item {\bf Frite os Bolinhos}
\begin {itemize}
\item Aque\c{c}a o óleo de canola puro em uma frigideira. Adicione suficiente óleo para que os bolinhos flutuem no óleo.
\item Usando duas colheres de sopa, molde os bolinhos e coloque no óleo quente.
\item Cozinhe até ficarem dourados.
\item Você irá cozinhá-los em vários lotes
\end {itemize}
\item {\bf Flavoring 1: Crisp Bacon}
\begin {itemize}
\item Cozinhe o bacon em uma frigideira, certificando-se de não cozinhar demais.
\item Depois do bacon estar frito, deixe escorrer as toalhas de papel.
\item Corte em pequenos pedaços e adicione à massa.
\end {itemize}
\item {\bf Flavoring 2: ervas frescas}
\begin {itemize}
\item Adicionar duas colheres de sopa de queijo parmesão ralado e cortar uma variedade de ervas frescas:
\begin {itemize}
\item Opção 1: cebolinha, salsa e manjericão
\item Opção 2: Alecrim e sálvia
\item Opção 3: duas colheres de sopa de cebolinha verde e tomilho finamente picados.
\end {itemize}
\item Sirva os bolinhos quentes como um aperitivo.
\end {itemize}
\end {enumerate}
\end {description}
\end {document}
