\documentclass [11pt, letterpaper] {article}
\input {headings}
\newcommand \recipeName {Pastel\~ao de Frango}
\chead {\recipeName}

\begin {document}
\input {title}

Pastel\~ao de frango era um prato que minha m\~ae sempre gostou de fazer e que eu e meu amigo Hor\'acio ador\'avamos comer na sexta \`a noite com uma cerveja. Servido quente \'e uma del\'icia. Em uma viagem a Santiago de Compostela em 2017, eu descobri que as origens do pastel\~ao de frango \'e a tradicional Empanada Gallega. A minha receita \'e bem modificada tanto da minha m\~ae quanto da empanada Gallega. 

\begin {description}

\item [Ingredientes:] \ \\
        \begin {itemize}
        \item 1 \href{PuffyPastry.html}{Receita mestre de Julia Child para Puff Pastry} de ``Dominar a Arte da Culinária Francesa - Volume Dois" (o folheado comprado na loja funcionaria bem.
        \item 1 \href{PastryDough.html}{Receita mestre de Julia Child para massa de torta} (pastry dough) do livro de ``The Way to Cook"
        \item 1 \href{ChickenFilling.html}{Receita Mestre da Mãe de Nelson para Recheio de Frango}
        \item 1 ovo
        \item Flour for dusting
        \item Óleo de cozinha sem sabor, manteiga ou spray para cozinhar para a assadeira
        \item 1 pitada de sal
        \end {itemize}

\item[Equipamento:] \ \\
        \begin {itemize}
        \item Forma plana de cor clara 
        \item Rolo de Massa
        \item Raspador de massa (dough scraper)
        \item L\^amina Gillete limpa
        \item Pincel de cozinha 
        \item Cortador de pizza rotativo pequeno para cortar e recortar a massa
        \end {itemize}

\item[Procedimento:]\ \\
\begin{enumerate}
\item [Estique a massa de baixo:] \ \\
        \begin{itemize}
        \item Unte uma forma grande e pulverize com farinha, reserve.
        \item Estique a massa de pastelaria refrigerada com um rolo até cobrir o
        superfície inferior inteira da forma.
        \item Role a massa no rolo e desenrolar em cima da forma
        \item Ajuste as arestas para se certificar de que apenas o fundo está coberto. A massa n\~ao deve subir aos lados da assadeira.
        \item Usando uma régua, divida a forma em 3
        ou 4 tiras longas (cada tira deve ser de 6 a 10 centimetros de largura).
        \end{itemize}
\item{\bf Espalhe o recheio:}    
       \begin{itemize}    
        \item Com uma colher pequena, coloque o recheio de frango em cima de cada tira. O recheio deve cobrir de 3 a 5 cent\'imetros do centro  de cada tira.
        \item Coloque a forma na geladeira enquanto trabalha
        com a massa folhada.
        \end{itemize}
\item{\bf Estique e corte a massa de cima:}    
       \begin{itemize}
        \item Estique a massa folhada refrigerada, com um rolo de massa, at\'e obter as mesmas
        dimens\~oes da forma (você pode usar uma segunda
        forma do mesmo tamanho que um guia). É mais fácil rolar se
        Você começa com uma forma retangular que tem o mesmo
        proporções como a assadeira. Certifique-se de usar uma abundância de
        farinha para esticar a massa folhada. Enquanto rola e levanta o
        pastelaria do balc\~ao (diminuirá um pouco quando você
        lift). Pare de rolar somente quando a massa descontraída estiver no tamanho
        da assadeira. 
        \item Se a massa ficar muito quente e macia,
        Você pode enrolá-lo em seu rolo e colocá-lo em um
        parafusos revestidos de papel em pergaminho e colocá-lo no
        refrigerador até esfriar e firmar para que você possa
        trabalho contínuo.
        \item Quando a massa folhada estiver esticada, corte as arestas,
        usando uma régua divida a massa no mesmo número de
        tiras que você dividiu o enchimento na assadeira com
        a massa de pastelaria.
        \begin {itemize}
                \item Faça pequenos cortes em ambos os lados da massa para
                marque os cortes.
                \item Use a borda da assadeira como guia para
                corte as tiras.
        \end {itemize}
        \end{itemize}
\item{\bf Cubra com a massa de cima:}    
       \begin{itemize}   
\item Traga a assadeira com a massa de massa e frango
        preenchendo a geladeira.
        \item Despeje água em uma tigela pequena.
        \item Usando a ponta dos dedos ou uma escova de pastelaria,
        escove uma tira de 1/2 polegada de água ao longo de ambos os lados do primeiro
        Frango de enchimento de galinha.
        \item Dobre suavemente uma das tiras de massa de massa ao meio.
        \item Levante a tira de massa e coloque cuidadosamente a metade de
        uma tira de enchimento de frango.
        \item Desdobre a tira de massa para cobrir toda a tira de enchimento.
        \item Com o dedo seco, pressione suavemente as bordas do sopro
        Pastelaria ao redor das tiras de enchimento de frango.
        \item Repita com outras tiras.
        \item Deve haver uma faixa estreita de massa de pastelaria descoberta
        entre, e talvez nas bordas externas, de cada tira de
        enchimento coberto de frango. Faça longos cortes retos ao longo da
        borda da massa folhada e remova esta massa extra para que
        Você tem bordas limpas para cada tira.

        \item Cubra a assadeira com papel de plástico e coloque
        na geladeira por pelo menos uma hora (você pode refrigerar por at\'e
        24 horas).
        \end {itemize}


\item [Asse o Pastel\~ao:] \ \\
        \begin {itemize}
        \item Posicione um rack na parte superior do meio do forno,
        e pré-aquecer o forno para 410 F.
        \item Bata um ovo e uma pitada de sal em uma tigela pequena até
        é bem agitado, mas não espumoso.
        \item Usando um pincel de massa, pincele o topo de cada
        tira de massa cheia com o ovo batido.
        \item Usando a l\^amina Gillete, faça pequenos 
        cortes (de até 3 cent\'imetros de comprimento) uniformemente espaçados no topo das tiras. Esses cortes são
        onde você cortará as tiras depois de serem assadas, assim
        faça-os do tamanho que voc\^e quiser servir.
        \item Se você decidiu fazer cada segmento de seis cent\'imetros ou mais,
        em seguida, faça um segundo pequeno corte diagonal em cima de cada segmento.
        \item Assa em um forno quente (400 F) até que a massa tenha inchado e tenha
        atingiu uma bela cor dourada (cerca de 40 minutos). Para melhores resultados, gire a assadeira após 25 minutos.
        \item Deixe esfriar por 15 minutos.
        \item Usando uma boa faca de pão, corte ao longo da perpendicular.
        cortes que você fez antes de assar.
        \item Coloque os pastéis em um rack para esfriar.
        \item Pode ser servido quente ou a temperatura ambiente.
        \end {itemize}
\end {enumerate}
\end {description}
\end{document}