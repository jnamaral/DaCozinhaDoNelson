\documentclass [11pt, letterpaper] {article}
\input {headings}
\newcommand \recipeName {Arroz de Carreteiro}
\chead {\recipeName}

\begin {document}
\input {title}

Arroz de Carreteiro é um prato tradicional do estado do Rio Grande do Sul. Provavelmente há tantas receitas para Arroz Carreteiro quanto há cozinheiros gaúchos. Aqui é um que eu uso. Eu faço o meu próprio ``charque" porque vivo em Edmonton, Alberta, mas no Rio Grande do Sul você pode comprar o charque já feito.

\vspace {0.3in}

\begin {description}

\item [Ingredientes:] \ \\
\begin {itemize}
\item 1 quilo de picanha
\item Sal grosso
\item Barbante de cozinha
\item 3 colheres de sopa de óleo de cozinha
\item 1 xícara de cebolas picadas
\item 2 xícaras de tomates descascados (picados)
\item 3 xícaras de arroz
\item 6 xícaras de água quente
\end {itemize}

\item [Procedimento:] \ \\

\begin {enumerate}
\item {\bf Faça o Charque (alguns dias antes de planejar o Carreteiro)}
\begin {itemize}
\item Cortar e retirar gordura e tecido sinuoso da carne.
\item Gentilmente passe sal grosso na carne até ficar coberto de sal.
\item Faça uma pequena fenda perto de uma extremidade de cada pedaço de carne com uma faca de aparar.
\item Passe um longo pedaço de barbante e amarre.
\item Amarre o fio para algo alto e coloque um grande recipiente embaixo dos pedaços de carne.
\item Deixe a carne salgada pendurada por alguns dias. Verifique todos os dias para garantir que a carne não esteja ficando seca demais (não deve ficar muito dura).
\item Quando a carne secou o suficiente e adquiriu uma cor roxa escura, você pode colocar a carne em sacos de plástico e armazenar na geladeira por até algumas semanas, ou no congelador por meses.
\end {itemize}
\item {\bf Pique o charque}
\begin {itemize}
\item Enxágüe o charque com água fria e remova qualquer sal visível que possa estar preso ao charque.
\item Dice o charque em peças muito pequenas (cerca de 1/4 de polegada)
\end {itemize}
\item {\bf Sautee e ferver o arroz}
\begin {itemize}
\item aqueça a água no microondas até que ferva --- quatro a cinco minutos.
\item Em uma grande panela pesada, aquecer um pouco de óleo de cozinha.
\item Sautee o charque, mas não o deixe ficar muito crocante.
\item Adicione as cebolas picadas e frite até ficarem macias.
\item Adicione os tomates picados e cozinhe até que a maior parte do líquido tenha evaporado.
\item Adicione o arroz e mexa-o até que alguns dos grãos começam a ficar opacos.
\item Adicionar a água quente.
\item Prove a água para ver se você precisa adicionar mais sal.
\item Reduza o fogo para o fogo lento mais lento do fog\~ao.
\item Despeje a água quente sobre o arroz mexendo.
\item Cubra a panela e cozinhe por cerca de 35 minutos. Não abra o pote.
\item Após 35 minutos, experimente alguns grãos da parte superior para ver se eles estão quase cozidos. Se estiverem, apague o fogo e mantenha o vaso fechado por mais dez minutos.
\item Não mexa o arroz.
\end {itemize}

\end {enumerate}
\end {description}
\end {document}
