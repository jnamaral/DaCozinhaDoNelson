
\documentclass [11pt, letterpaper] {article}
\input {headings}
\newcommand \recipeName {Ajoblanco}
\chead {\recipeName}

\begin {document}
\input {title}

Esta receita vem de Málaga no sul da Espanha, no Nordeste de Gibraltar. Ajoblanco de Málaga \'e muito tradicional. Ele tamb\'em é conhecido como Gazpacho Branco Espanhol. É uma sopa cremosa e branca. Os sabores mais prominentes em Ajoblanco são amêndoas, azeite e alho. A receita tradicional apresenta alho cru que daria um sabor de alho assertivo prominente. Não consigo tolerar esse sabor. Assim, minha adaptação é assar o alho suavemente no azeite para obter um sabor de alho muito mais suave e mais doce - os cidadãos de Málaga provavelmente protestariam por essa mudança, mas essa é minha herança brasileira em ação.


\begin {description}

\item [Ingredientes:] \ \\
\begin {itemize}
\item 6 fatias de pão sanduíche branco saudável, crostas removidas
\item 4 xícaras de água
\item 2 1/2 xícaras (8 3/4 onças) mais 1/3 xícara de amêndoas em pedaços em fatias
\item 2 dentes de alho descascados
\item 3 colheres de sopa de vinagre de xerez
\item Kosher sal e pimenta
\item Pinch pimenta caiena
\item 1/2 xícara de azeite extra virgem, mais um pouco para perfumar a sopa
\item 1/8 colher de chá de extrato de amêndoa
\item 2 colheres de chá de óleo vegetal
\item 150 gramas de uvas verdes sem sementes, cortadas em fatias finas (1 xícara)
\end {itemize}


\item [Procedimento:] \ \\

\begin {enumerate}
\item {\bf Assado o alho}
\begin {itemize}
\item Coloque duas colheres de sopa de azeite na sua panela mais pequena.
\item Corte os dentes de alho na metade e adicione à panela.
\item Coloque sobre um fogo muito lento e cozinhe até o alho virar uma cor loura pálida.
\item Coloque imediatamente o azeite e os dentes de alho em um prato pequeno e frio e reserve.
\end {itemize}

\item {\bf Molhe o pão}
\begin {itemize}
\item Combine o pão e a água na tigela e deixe embeber por 5 minutos.
\end {itemize}

\item {\bf Processe Amêndoas e Pão}
\begin {itemize}
\item Processe 2 1/2 xícaras de amêndoas no processador de alimentos até finamente moído, cerca de 30 segundos, raspando os lados do processador conforme necessário.
\item Usando as mãos, retire o pão da água, aperte-o levemente e transfira para o processador de alimentos com amêndoas. Reserve a água.
\item Adicione o alho assado com seu óleo, vinagre, 1 1/4 colheres de chá de sal e caiena para liquidar e processar até a mistura ter consistência de massa de bolo, 30 a 45 segundos.
\end {itemize}


\item {\bf Adicione óleo de oliva}
\begin {itemize}
\item Com o processador em execução, adicione o óleo de oliva em fluxo fino e estável, cerca de 30 segundos.
\item Adicionar água de imersão reservada e processar por 1 minuto.
\item Tempere com sal e pimenta a gosto.
\end {itemize}

\item {\bf Coe a sopa}
\begin {itemize}
\item Coe a sopa em um coador de malha fina ajustado sobre uma tigela, pressionando sólidos para extrair o líquido.
\item Descarte os sólidos.
\end {itemize}

\item {\bf Adicione Extracto de Amêndoa}
\begin {itemize}
\item Me\c{c}a 1 colher de sopa da sopa na segunda tigela e misture o extrato de amêndoa.
\item Retorne 1 colher de chá de mistura de extrato para sopa; descarte o restante.
\item Refrigere por pelo menos tr\^es horas ou até 24 horas.
\end {itemize}

\item {\bf Toste as am\^endoas}
\begin {itemize}
\item Forno quente a 300 F.
\item Misture 1/3 de xícara de amêndoas fatiadas com o óleo vegetal e espalhe-as sobre uma assadeira de cor clara.
\item Amêndoas torradas, mexendo de vez em quando, até serem levemente torradas.
\item Remova e deixe esfriar.
\end {itemize}

\item {\bf Servir a Sopa}
\begin {itemize}
\item Sopa de lenço em tigelas rasas.
\item Monte uma quantidade igual de uvas no centro de cada tigela.
\item Polvilhe amêndoas resfriadas sobre sopa e perfume com azeite extra. Sirva imediatamente.
\end {itemize}

\end {enumerate}
\end {description}
\end {document}
