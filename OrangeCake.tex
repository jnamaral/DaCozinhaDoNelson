\documentclass [11pt, letterpaper] {article}
\input {headings}
\newcommand \recipeName {Bolo de Laranja Inteira}
\chead {\recipeName}

\begin {document}
\input {title}

Este é outro bolo sem leite. Recebi esta receita da minha mãe no
Brasil. A receita original processa uma laranja inteira, com a casca e
tudo, e usa duas xícaras de açúcar. Muito provável para compensar pelo
amargor da casca branca. Descobri que, com apenas um pouquinho mais de
trabalho para ralar a laranja e remover a pele branca, pode-se reduzir a quantidade de açúcar.

A massa é bastante líquida, por isso quando eu asso em uma forma tubular com
centro removível eu coloco a forma dentro de uma assadeira para evitar sujar o forno.

Em 2017, eu decidi pesquisar a cozinha Grega e descobri uma receita
para um antigo bolo de farinha de semolina e laranja que também usa
duas laranjas inteiras, mas naquela receita as laranjas s\~ao lavadas primeiro. Eu me pergunto se esta receita brasileira é uma versão tropical da antiga receita Grega.

\begin {description}

\item [Ingredientes:] \ \\
\begin {itemize}
\item 1 laranja inteira
\item 1 1/2 xícara de açúcar (300 grams)
\item 1/2 colher de chá de sal
\item 4 ovos
\item 1 xícara de óleo de cozinha sem sabor (210 grams)
\item 2 colheres de sopa de fermento em pó
\item 2 1/2 xícaras de farinha (354 grams)
\item 1 xícara de suco de laranja com açucar
\end {itemize}


\item [Procedimento:] \ \\
\begin {enumerate}
\item {\bf Pré-aqueça o forno e prepare a forma}
\begin {itemize}
\item Pré-aqueça o forno para 175 C.
\item Coloque uma assadeira limpa no forno (você pode cobri-la com papel alumínio para evitar derramamentos na assadeira).
\item Unte uma forma tubular com centro removível com margarina ou
\'oleo sem sabor.
\end {itemize}
\item {\bf Prepare a laranja}
\begin {itemize}
\item Coloque o açúcar em um processador de alimentos.
\item Rale a pele amarela de toda  a laranja, sem ralar a parte
branca, e adicione o sumo ralado ao açúcar.
\item Processe por uns 30 segundos.
\item Remova a casca branca da laranja e descarte.
\end {itemize}
\item {\bf Termine a massa}
\begin {itemize}
\item Adicione o sal à mistura de açúcar e sumo.
\item Adicione a laranja descascada e processe at\'e obter uma pasta l\'iquida.
\item Adicione os ovos à mistura de açúcar.
\item Processe por um minuto.
\item Derrame lentamente o óleo através do tubo de alimentação do processador de alimentos com o processador funcionando.
\item Usando uma espátula de borracha, transfira o conteúdo do processador de alimentos para uma tigela grande.
\item Usando um coador grande, penere a farinha e o fermento na
mistura molhada e incorpore. O melhor é incorporar a farinha em três bateladas para evitar caro\c{c}os..
\end {itemize}
\item {\bf Bake}
\begin {itemize}
\item Remova a assadeira quente do forno, coloque a forma tubular untada no centro.
\item Coloque a massa na forma untada e transfira, com a assadeira embaixo, para o forno.
\item Asse por cerca de 45 minutos ou até um palito inserido no centro sair muito limpo.
\item Remova do forno, coloque sobre uma assadeira e despeje o suco de laranja em cima do bolo quente.
\item Deixe descansar por 5 minutos.
\item Passe uma faca limpa e afiada ao redor da borda da panela do tubo e levante o centro para remover a parte externa da bandeja do tubo.
\item Coloque o bolo, com o centro da forma tubular ainda encaixado, em uma grade para esfriar.
\item Deixe esfriar por 30 minutos antes de remover o centro da forma.
\end {itemize}
\end {enumerate}
\end {description}

\begin{table}
\begin{tabular}{cccc}
\includegraphics[width=0.25\textwidth]{\imageDir/OrangeCake/IMG_8585.jpg} &
\includegraphics[width=0.25\textwidth]{\imageDir/OrangeCake/IMG_8586.jpg} &
\includegraphics[width=0.25\textwidth]{\imageDir/OrangeCake/IMG_8587.jpg} &
\includegraphics[width=0.25\textwidth]{\imageDir/OrangeCake/IMG_8588.jpg} \\
\includegraphics[width=0.25\textwidth]{\imageDir/OrangeCake/IMG_8589.jpg} &
\includegraphics[width=0.25\textwidth]{\imageDir/OrangeCake/IMG_8608.jpg} &
\includegraphics[width=0.25\textwidth]{\imageDir/OrangeCake/IMG_8640.jpg} &
\includegraphics[width=0.25\textwidth]{\imageDir/OrangeCake/IMG_8641.jpg} \\
\includegraphics[width=0.25\textwidth]{\imageDir/OrangeCake/IMG_8642.jpg} \\
\end{tabular}
\end{table}

\end {document}
