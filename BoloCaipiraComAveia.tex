\documentclass[11pt,letterpaper]{article}
\input{headings}
\newcommand \recipeName {Bolo Caipira Com Aveia}
\chead{\recipeName}

\begin{document}
\input{title}

Minha m\~ae fez este bolo quando est\'avamos visitando meu irm\~ao em Bras\'ilia. Ficou muito bom e eu pedi a receita. Muitas varia\c{c}oes sao poss\'iveis para fazer ele sem gluten (veja a nota sobre pulverizar a forma), sem leite (substitua o leite por leite de am\^endoa --- normalmente pessoas com intoler\^ancia a lactose podem consumir queijo parmes\~ao),  vegetariana (use tomates secos conservados em \'oleo no lugar do frango), peixeriana (use atum), ou com bacon frito no lugar do frango.
 
\begin{description}

\item[Ingredients:]\ \\
	\begin{itemize}
	\item 4 ovos
	\item 1 lata de milho (200 gramas) em temperatura ambiente
	\item 1/4 de x\'icara de oleo de canola (50 ml)
	\item 1 x\'icara de leite (250 ml)
	\item 1 colher de cha de sal
	\item 1/2 colher de sopa de oregano seco
	\item 1 x\'icara de cheiro verde (salsinha e cebolinha)
	\item 1 x\'icara de aveia em flocos
	\item 50 grams de parmes\~ao ralado
	\item 1 colher de sopa de fermento em p\'o
	\item 100 gramas de peito de frango cozido com sal e desfiado (pode usar atum ou tomates secos em oleo)
	\item 2 tomates italianos cortados em cubos (220gr)
	\item Oleo e farinha de p\~ao para untar e enfarinhar a forma
	\end{itemize}

\item[Procedure:]\ \\
	\begin{enumerate}
	\item {\bf Pre-aque\c{c}a of forno e prepare a forma.}
	\begin{itemize}
	\item Ligue o forno a 200 C.
	\item Unte muito bem a forma com \'oleo
	\item Pulverize muito bem com a farinha de p\~ao. Para uma vers\~ao sem gluten, pulverize um pouco de aveia no processador de comida para fazer um farelo e pulverize a forma com este farelo no lugar do farelo de p\~ao.
	\end{itemize}
	
	\item {\bf Processo da Aveia}
	\begin{itemize}
	\item Coloque a aveia na tigela do processador de alimentos e pulso cinco ou seis vezes at\'e obter flocos grosseiros.
	\item Deslize os flocos na tigela grande.
	\end {itemize}

	\item {\bf Processe of ingredientes molhados}
	\begin{itemize}
	\item Num processador de alimentos, ou liquidificador, processe os ovos, o milho, o oleo, o leite e o sal ate' obter uma massa homog\^enea.
        \item Transfira a mistura para a tijela que cont\'em os flocos de aveia.
	\end{itemize}
	\item {\bf Adicione os ingredientes secos}
	\begin{itemize}
	\item Adicione o oregano, o cheiro verde, a aveia, o parmes\~ao ralado e o fermento e misture.
	\item  Adicione o frango e tomates, ou o atum, ou os tomates secos (se usar tomates secos, nao use o tomate fresco).
	\end{itemize}
	\item {\bf Asse}
	\begin{itemize}
	\item Coloque a mistura na forma e leve ao forno pre-aquecido (200 C) e asse por aproximadamente 40 minutos.
	\item O bolo est\'a assado quando um palito inserido no centro sair limpo.
	\item Retire do forno e deixe esfriar por 5 minutos.
	\item Passe uma faca estreita de ponta ao redor da forma e ao redor do centro.
	\item Coloque o prato onde vai servir em cima da forma e inverta a forma para desenformar o bolo.
	\item Sirva ainda quente.
	\end{itemize}
	\end{enumerate}
\end{description}
\end{document}



