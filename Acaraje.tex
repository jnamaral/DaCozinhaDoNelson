\documentclass [11pt, letterpaper] {article}
\input {headings}
\newcommand \recipeName {Acaraj\'e}
\chead {\recipeName}

\begin {document}
\input {title}

Acaraj\'e é um prato da Bahia no Nordeste do Brasil. As raízes são africanas. É um bolinho de massa frito que é feito apenas com ervilhas de olhos pretos temperadas com cebolas e fritas no óleo de dend\^e. Em Edmonton, você encontra feij\~ao fradinho descascado na Loja Africana do Excel.

\vspace {0.3in}

\begin {description}

\item [Ingredientes:] \ \\
\begin {itemize}
\item 500 gramas de feij\~ao fradinho descascado
\item 1 cebola média
\item 1 colher de chá de sal
\item 3 xícaras de dend\^e óleo
\end {itemize}

\item [Procedimento:] \ \\

\begin {enumerate}
\item {\bf Coloque o feij\~ao fradinho de molho}
\begin {itemize}
\item Coloque o feij\~ao fradinho em uma tigela funda e cubra com muita água.
\item Mexa bem e deixe o feij\~ao se acomodar no fundo da tigela.
\item Usando um filtro, remova todas as cascas que flutuam para o topo.
\item Continue mexendo e removendo cascas até que a maioria desapareça.
\item Escorra a \'agua usando um coador grande para separar o feij\~ao fradinho.
\item Repita a lavagem do feij\~ao duas vezes mais.
\item Deixe o feij\~ao fradinho de molho durante a noite.
\end {itemize}

\item {\bf Prepare a massa}
\begin {itemize}
\item Descasque e corte a cebola em peda\c{c}os grandes.
\item Coloque cebola picada na tigela de um grande processador de alimentos ou liquidificador e processe até que seja um purê líquido.
\item Escorrer o feij\~ao fradinho muito bem e adicionar ao processador de alimentos, ou liquidificador, e processar até obter uma massa lisa.
\item Adicionar as cebolas picadas e o sal.
\item Processe para incorporar bem.
\item Transfira a massa para a tigela de uma batedeira. 
\item Processe na batedeira, com uma pá, por vários minutos até a mistura dobrar em volume
\end {itemize}

\item {\bf Frite os Acaraj\'es}
\begin {itemize}
\item Despeje o óleo de dend\^e em uma frigideira de fundo pesado --- você pode adicionar óleo não aromatizado, como Canola se você não tiver dend\^e suficiente.
\item Aque\c{c}a o óleo até ficar moderadamente quente (350 F).
\item Usando duas colheres grandes forme grandes aglomerados da massa e baixe no óleo quente.
\item Acaraj\'e deve fritar em óleo moderadamente quente por um tempo bastante longo --- o tempo exato depende do tamanho do acaraj\'e.
\item Uma vez que obteve cor em um lado, vire o acaraj\'e. Continue girando de vez em quando até uma cor de mogno profundo.
\item Sirva imediatamente depois de fritar com vatap\'a e um molho feito de tomate, cebola, lima, coentro e pimentas.
\end {itemize}

\end {enumerate}
\end {description}
\end {document}
