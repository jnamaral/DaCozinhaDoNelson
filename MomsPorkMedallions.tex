\documentclass [11pt, papel de carta] {article}
\input {headings}
\newcommand \recipeName {Medalhões de porco da minha mãe}
\chead {\recipeName}

\begin {document}
\input {title}

\begin {flushright}
Esta receita é retirada, com algumas adaptações, do livro da minha mãe, Dioraci Rambo Urtassum, livro de receitas {\it Delícias, aromas e vidas}.
\end {flushright}

\begin {description}

\item [Ingredientes:] \ \\
\begin {itemize}
\item 1 kg de lombo de porco
\item 2 colheres de chá de sal
\item 1/2 colheres de chá de canela em pó
\item 1/4 colher de chá de cravo moído
\item 1 dente de alho
\item 1 pimenta de cheiro ou 4 gotas de molho de pimenta vermelha
\item 1 colher de sopa de molho de soja
\item 1/2 xícara de água
\item 4 colheres de sopa de manteiga
\item 2 colheres de sopa de óleo de cozinha
\item 2 colheres de sopa de alecrim picado
\item 1 colher de sopa de amido de milho
\end {itemize}


\item [Procedimento:] \ \\
        \begin {description}
        \item [Marinar a carne de porco] \ \\
                \begin {itemize}
\item Limpe os lombos, removendo a pele de nervos e toda a gordura.
\item Corte as pontas finas dos filetes e reserve para outro uso.
\item Fatie os filetes em fatias grossas de 4 a 5 centímetros.
\item Esmagar o alho com o sal usando uma faca de chef em uma tábua de madeira para obter uma pasta.
\item Se usar a pimenta o cheiro, esmague a pimenta na pasta de alho.
\item Caso contrário, adicione as gotas de molho de pimenta na pasta de alho e misture bem para incorporar.
\item Esfregue a pasta de alho nos pedaços de lombo de porco.
\item Misture a canela em pó com o cravo moído e pulverize sobre a carne de porco.
\item Adicione o 1/2 xícara de água, cubra e coloque na geladeira por pelo menos uma hora. Isso pode ser feito várias horas antes do tempo.
\end {itemize}
\item [Secar a carne de porco] \ \\
\begin {itemize}
\item Coloque uma grade sobre uma forma grande.
\item Esmagar cada pedaço de carne de porco com a palma da mão contra uma tábua de cortar para criar medalhões mais finos e mais largos.
\item Coloque cada medalhão no topo do rack deixando algum espaço entre eles.
\item Coloque, descoberto na geladeira por pelo menos umas duas horas para deixar os pedaços de carne de porco secar um pouco. Desta forma, eles vão dourar melhor depois.
\end {itemize}
\item [Dore a carne de porco] \ \\
\begin {itemize}
\item Escolha uma panela de fundo pesado com uma tampa que encaixe bem. Eu uso uma frigideira funda de aço inoxidável. Adicione 2 colheres de sopa de manteiga e óleo de cozinha e aqueça até a manteiga derreter e começar a corar.
\item Coloque pedaços de carne de porco na panela, sem amontoar. Não mova as peças depois de colocá-las. Deixe cozinhar em fogo alto até o primeiro lado ficar dourado.
\item Vire as peças e marrons o segundo lado.
\item Tenha cuidado para não cozinhar demais. Os medalhões ainda devem ter um anel de carne crua muito rosa no meio.
\item Se necessário, dore os pedaços de carne de porco em lotes, mantendo-os em um prato coberto aquecido quando estiverem dourados.
\item Uma vez que você tenha terminado com todos os pedaços de carne de porco, coloque todos eles na panela, remova a panela do queimador quente em um frio, cubra bem e deixe o assento, coberto, por poucos minutos.
\item Enquanto isso, coloque seu prato de servir em um forno a 200 F para aquecê-lo.
\end {itemize}
\item [Finalize o molho e sirva] \ \\
\begin {itemize}
\item Misture o amido de milho com duas colheres de sopa de água em uma tigela pequena ou xícara até dissolver.
\item Retire os medalhões de porco para o prato de servir, cubra e mantenha quente.
\item Adicione o molho de soja à panela.
\item Raspe todos os pedaços dourados que grudaram na panela com o molho.
\item Adicione o amido de milho dissolvido na panela.
\item Coloque a panela em fogo moderadamente alto e cozinhe mexendo até que o molho engrosse.
\item Retire do fogão e junte as duas colheres de sopa de manteiga e o alecrim picado, mexendo sempre até a manteiga se dissolver.
\item Despeje o molho sobre os medalhões de porco e sirva.
\end {itemize}
\end {description}
\end {description}
\end {document}
%