\documentclass [11pt, letterpaper] {article}
\input {headings}
\newcommand \recipeName {Romeu e Julieta}
\chead {\recipeName}

\begin {document}
\input {title}

Na década de 1970, Maur\'icio de Sousa, o cartunista brasileiro mais bem sucedido e criador da "Turma da Monica" foi contratado para criar uma campanha publicitária para "Goiabada da CICA". Goiabada é uma pasta de goiaba doce que é popular no Brasil. A campanha apresentou a goiabada muito doce combinada com um queijo fresco salgado. A ideia de Mauricio era chamar a Goiabada doce "Julieta" e o queijo salgado "Romeu". A sobremesa "Romeu e Julieta" ficou muito popular e este nome hoje \'e conhecido em todo o Brasil. Aqui temos em uma reinterpretação do website do glitztv ~ \footnote {http://www.glitztv.com.br/noticias-br/romeu-e-julieta-ganha-versao-com-queijo-mascarpone/} com base feita de queijo mascarpone.

\vspace {0.3in}

\begin {description}

\item [Ingredientes:] \ \\
\begin {itemize}
\item 200 gramas de goiabada
\item 200 ml de água
\item 200 gramas de queijo mascarpone
\item 200 gramas de creme pesado
\item 1 grande pitada de sal
\item amêndoas fatiadas e torradas para decoração
\end {itemize}

\item [Procedimento:] \ \\

\begin {enumerate}
\item {\bf Preparando o goiabada}
\begin {itemize}
\item Corte o goiabada em pequenos cubos e coloque uma panela m\'edia de fundo grosso.
\item Adicione 200 ml de água.
\item Cozinhe em fogo lento, mexendo de vez em quando até obter um molho brilhante e homogêneo.
\item Deixe esfriar à temperatura ambiente.
\end {itemize}

\item {\bf Toste as amêndoas fatiadas}
\begin {itemize}
\item Toste as amêndoas fatiadas em um forno de 150 C observando freqüentemente para que elas fiquem uma cor dourada clara, mas não se tornem amargas.
\end {itemize}

\item {\bf Prepare o mousse}
\begin {itemize}
\item Coloque o queijo mascarpone, o creme de leite e o sal na tigela de uma batedeira.
\item Bata at\'e obter a consistência de mousse.
\item Prove o sal, este mousse de queijo deve ser levemente salgado para contrastar com a goiabada doce.
\end {itemize}

\item {\bf Servir}
\begin {itemize}
\item Sirva em ta\c{c}as de martini com o goiabada no fundo da ta\c{c}a e uma grande colher da mousse flutuando nela.
\item Decore com as amêndoas torradas torradas.
\end {itemize}

\end {enumerate}
\end {description}
\end {document}

