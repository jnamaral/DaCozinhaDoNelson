\documentclass [11pt, letterpaper] {article}
\input {headings}
\newcommand \recipeName {Massa de Tortilha do Daniel}
\chead {\recipeName}

\begin {document}
\input {title}

Eu desenvolvi essa receita como uma refeição rápida para Daniel. Com o tempo, tornou-se um alimento básico em nossa casa. O sabor do azeite de oliva, do queijo parmesão e da pimenta preta fazem a tortilha adquirir um sabor de massa. Uma vez que você tenha o \href {DanielsMeatSauce.html} {Molho de Carne de Daniel} no congelador --- veja a receita --- esta é uma refeição que você pode preparar em minutos.

\vspace {0.3in}
\begin {description}

\item [Ingredientes:] \ \\
\begin {itemize}
\item 150 gramas de \href {DanielsMeatSauce.html} {Molho de Carne de Daniel}
\item 2 tortilhas de farinha
\item 1 colher de sopa de azeite virgem extra
\item 2 colheres de sopa de queijo parmesão recentemente ralado
\item Pimenta do reino recém-moída
\end {itemize}

\item [Procedimento:] \ \\
\begin {itemize}
\item Aquecer o molho de carne no forno de microondas até que seja aquecido até vapor estar saindo do prato.
\item Corte a tortilha em tiras finas como mostrado nas imagens abaixo.
\item Solte, com os dedos, as tiras que ficaram juntas durante o corte em uma tigela.
\item Salpique com o azeite e misture bem.
\item Espalhe o queijo parmesão recentemente ralado e a pimenta recém-mo\'iada e mexa suavemente.
\item Verifique se o molho de carne ainda está cozinhando, se não colocar novamente no microondas por mais um minuto.
\item Monte o prato em dois pratos fundos aquecidos, colocando uma pequena camada de "massa de tortilha" e depois um pouco de molho, repetindo as camadas três vezes em cada tigela.
\item Coloque cada prato fundo no microondas por 30 segundos em plena potência e sirva imediatamente. Estes 30 segundos apenas aquecerão as tiras de tortilha sem torná-las duras.
\end {itemize}
\end{description}

\begin{table}
\begin{tabular}{cccc}
\includegraphics[width=0.25\textwidth]{../FromNelsonsKitchen/TortillaPasta_images/IMG_8300.jpg} &
\includegraphics[width=0.25\textwidth]{../FromNelsonsKitchen/TortillaPasta_images/IMG_8302.jpg} &
\includegraphics[width=0.25\textwidth]{../FromNelsonsKitchen/TortillaPasta_images/IMG_8303.jpg} &
\includegraphics[width=0.25\textwidth]{../FromNelsonsKitchen/TortillaPasta_images/IMG_8304.jpg} \\
\includegraphics[width=0.25\textwidth]{../FromNelsonsKitchen/TortillaPasta_images/IMG_8305.jpg} &
\includegraphics[width=0.25\textwidth]{../FromNelsonsKitchen/TortillaPasta_images/IMG_8306.jpg} &
\includegraphics[width=0.25\textwidth]{../FromNelsonsKitchen/TortillaPasta_images/IMG_8307.jpg} &
\includegraphics[width=0.25\textwidth]{../FromNelsonsKitchen/TortillaPasta_images/IMG_8308.jpg} \\
\end{tabular}
\end{table}
\end{document}


