
\documentclass [11pt, letterpaper] {article}
\input {headings}
\newcommand \recipeName {Ovos Cozidos Duros}
\chead {\recipeName}

\begin {document}
\input {title}

\begin {flushright}
{\bf De {\it The Way to Cook} por Julia Child}
\end {flushright}
 
 Cozinhar um ovo é considerada uma das tarefas mais triviais em uma cozinha. Muitas vezes, ouvimos a piada ``pelo menos ela sabe cozinhar um ovo" ou ``ele nem sequer pode cozinhar um ovo". Bem, a verdade é que, se você se importar com o resultado, ferver um ovo duro é uma tarefa de culinária simples mas notável precisa. E uma vez que você começa a prestar atenção como eu, você verá muitos restaurantes, alguns em hotéis caros, que falharam nessa tarefa. Um ovo duro perfeitamente fervido não terá nem o mais sutil escurecimento da pele da gema. O escurecimento da pele é devido à formação de enxofre quando a gema permanece por muito tempo em temperatura alta. Os ovos tamb\'em devem ser f\'aceis de descascar. O procedimento desta receita foi desenvolvido pela Georgia Egg Board e aparece no Julia Child's {\it The Way to Cook}.
 
\begin {description}

\item [Ingredientes:] \ \\
\begin {itemize}
\item Até quatro ovos
\item 2 litros de água
\item Se cozinhar 5 a 12 ovos use 3 1/2 litros de água
\end {itemize}


\item [Procedimento:] \ \\
\begin {enumerate}
\item {\bf Cozinhe os ovos}
\begin {itemize}
\item Você precisará de uma tigela grande com água gelada com gelo dentro no final deste tempo de cozimento.
\item Se voc\^e for servir os ovos inteiros e quer que eles tenham um bom formato, fa\c{c}a um furo com uma agulha na ponta mais grossa do ovo para o ar escapar. N\~ao perfure a membrana interna.
\item Coloque cuidadosamente os ovos na panela fria.
\item Cubra os ovos com água fria.
\item Coloque em fogo alto at\'e levantar a fervura.
\item Remova do fogo.
\item Cubra a panela e deixe fechada por exatamente 17 minutos.
\end {itemize}
\item {\bf Banho gelado de dois minutos}
\begin {itemize}
\item Transfira os ovos para a tigela de água gelada com gelos flutuando.
\item Coloque imediatamente a panela onde os ovos foram fervidos de volta ao fogo alto.
\item Deixe os ovos esfriarem na água gelada por exatamente dois minutos.
\end {itemize}
\item {\bf Fervura de dez segundos}
\begin {itemize}
\item Transferir alguns ovos de cada vez para a água que agora deve estar fervendo.
\item Deixe os ovos na água fervente por exatamente 10 segundos --- o motivo é expandir as cascas e facilitar descascar depois.
\item Quebre as cascas suavemente em vários lugares e submerja os ovos na água gelada.
\item Deixe os ovos na água gelada por 15 a 20 minutos antes de descascar
\end {itemize}
\item {\bf Descasque os ovos}
\begin {itemize}
\item Quebre bem a casca de ovo por toda parte, batendo ele suavemente contra o balcão.
\item Segure o ovo sob uma fina corrente de água.
\item Comece a descascar na extremidade maior.
\item Retorne cada ovo de volta para a água gelada assim que é descascado.
\end {itemize}
\item {\bf Armazenamento}
\begin {itemize}
\item Para armazenar, mantenha os ovos imersos em água fria na geladeira em uma vazilha descoberta.
\end {itemize}
\end {enumerate}
\end {description}
\end {document}
