\documentclass [11pt, letterpaper] {article}
\input {headings}
\newcommand \recipeName {Massa de Torta da Julia Child}
\chead {\recipeName}

\begin {document}
\input {title}

\begin {flushright}
De {\it The Way to Cook} por Julia Child
\end {flushright}

Esta é uma receita mestre. Os detalhes são importantes. O mais importante é não travalhar esta massa demais e não adicionar muita água. A massa est\'a pronta para descan\c{c}ar quanto ela estiver apenas formando uma bola irregular com pequenos pedaços de manteiga ainda vis\'iveis.
 
\begin {description}

\item [Ingredientes:] \ \\
\begin {itemize}
\item 1 1/2 xícaras de farinha 
\item 1/2 xícara de farinha de bolo
\item 1 colher de chá de sal
\item 170 gramas de manteiga sem sal em tablete fria
\item 4 colheres de sopa de gordura vegetal hidrogenada  ou banha de porco
\item 1/2 xícara de água gelada
\end {itemize}

\item [Procedimento:] \ \\
\begin {enumerate}
\item {\bf Corte a manteiga}
\begin {itemize}
\item Usando uma faca de l\^amina estreita, corte a manteiga fria em cubos de 1 cent\'imetro.
\end {itemize}
\item {\bf Se estiver usando of processador de alimentos}
\begin {itemize}
\item Adicione a farinha e o sal e pulse uma vez para misturar.
\item Adicione a manteiga fria cortada e pulse 5 ou 6 vezes para quebrar a manteiga.
\item Adicione a gordura vegetal ou a banha e o pulse até que se assemelhe a migalhas.
\item Adicione a água gelada pulsando até que comece a formar uma bola.  Não processe a massa demais.
\end {itemize}
\item {\bf Usando o cortador de massa\footnote{O cortador the massa \'e um utens\'ilio comum nos Estados Unidos e que simplifica o trabalho de cortar a gordura na farinha. O mesmo efeito pode ser obtido com um garfo.}}
\begin {itemize}
\item Misture a farinha e o sal com um garfo em uma tigela.
\item Adicione a manteiga e corte-a com o cortador de massa, um garfo ou a ponta dos dedos.
\item Adicione a gordura vegetal ou a banha e misture-o até se parecer com as migalhas. Voc\^e ainda deve poder ver pequenos pedaços de manteiga na massa.
\item Adicione a água um pouco de cada vez até que você tenha uma massa coesa que n\~ao esbugalhe quando pressionada na mão.
\end {itemize}
\item {\bf Forme e enrole a massa}
\begin {itemize}
\item Coloque a massa em uma superfície de trabalho e forme em quadrado (ou em dois discos se você vai usar para tortas).
\item Enrole em plástico e refrigere pelo menos 30 minutos, mas uma hora é preferível.
\item A massa pode ser mantida refrigerada por vários dias.
\end {itemize}
\end {enumerate}
\end {description}
\end {document}
