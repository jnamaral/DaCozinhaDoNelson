\documentclass [11pt, letterpaper] {article}
\input {headings}
\newcommand \recipeName {Mandioca Frita}
\chead {\recipeName}

\begin {document}
\input {title}

{\em Mandioca Frita} é um aperitivo tradicional servido ao longo do
Sul do Brasil. A mandioca recém-cozida com o vapor subindo
do prato é um prato muito popular para servir com carnes grelhadas.
A mandioca frita \'e geralmente preparada no dia seguinte ao cozimento da mandioca. A mandioca que encontramos em supermercados na Am\'erica do Norte
está encerada para preservação. Ao comprar, tente selecionar raízes
que não são feridas e que não têm manchas escuras. Compre sempre extra
porque você vai desperdiçar algum. Prefere a espessura do meio, como
espécimes muito grossos são lenhosos.

Quando colhida fresco do solo, a mandioca é muito fácil de descascar. Ela
tem uma casca dupla, uma casquinha muito fina marrom e uma pele branca mais grossa
embaixo. Você deve remover as duas peles. Na
mandioca rec\'em colhida, a pele branca se separa do centro da raiz.
sem esforço. Eu costumava ter muita dificuldade descascando a
mandioca encerada que encontramos na América do Norte porque a casca parece ser
firmemente grudada \`a raiz. Agora eu descobri que submergindo as raízes
por alguns minutos em água muito quente, a casca se expande e solta
da raiz, tornando-se quase tão fácil de descascar como as ra\'izes rec\'em
colhidas.

Uma alternativa muito mais fácil é comprar mandioca congelada que já foi descascada. Em Edmonton você pode encontrar mandioca congelada no Paraiso Tropical, SuperStore e às vezes em lojas chinesas, como o Lucky 97.

\begin {description}

\item [Ingredientes:] \ \\
\begin {itemize}
\item mandioca
\item Óleo de canola
\item Sal
\item Água
\item Maionese
\item Ketchup
\item Mostarda
\end {itemize}


\item [Procedimento:] \ \\
\begin {enumerate}
\item {\bf Peel the mandioca}
\begin {itemize}
\item Encha sua pia com água quente e submerja as ra\'izes de mandioca nesta água por alguns minutos para afrouxar a casca.
\item Corte as extremidades de cada raiz de mandioca com uma faca grande.
        \item Com uma faca de pequena e pontuda, descasque a mandioca começando no topo.
\item Corte a mandioca em segmentos de 8 a 10 centimetros de comprimento.
\item Suporte cada segmento e corte ao meio longitudinalmente.
\item Corte cada meio pela metade novamente.
\item Remova o fiapo central de cada segmento.
\item Remova quaisquer peda\c{c}os escuros ou fibrosos.
\item Neste ponto, você pode submergir a mandioca descascada em água doce para que todos os pedaços de mandioca sejam cobertos por água, e coloque no congelador, ou você pode prosseguir a cozinhar.
\end {itemize}
\item {\bf Cozinhe a mandioca}
\begin {itemize}
\item Se você congelou a mandioca, coloque a mandioca congelada com a água circundante em uma panela de água fresca e coloque a ferver.
\item Caso contrário, coloque a mandioca descascada numa panela de press\~ao e cubra com água. Não adicione nenhum tempero ou sal à água.
\item Cubra a panela de press\~ao e cozinhe, sob pressão, por 15 minutos. 
\item Coloque a panela de pressão sob água fria e remova cuidadosamente a válvula de pressão para liberar o vapor.
\item Desligue o fogo, adicione sal à água e deixe reposar por 5 a 10 minutos.
\item Escorre a mandioca cozida em uma coador de massa.
\item Muitas vezes, a mandioca cozida é servidos como um acompanhamento para pratos de carne.
\end {itemize}


\item {\bf Primeira fritura da mandioca}
\begin {itemize}
\item O melhor é fritar a mandioca quando secou bem depois de ser fervida. Normalmente, no Brasil, você frita a mandioca no dia seguinte ao que você ferveu. Para ela secar o bom e' deixar no refrigerador descoberta.
\item Aquecer o óleo de canola puro em uma panela. Use bastante óleo para submergir os pedaços de mandioca no óleo.
\item Adicione uma meia dúzia de peda\c{c}ps de mandioca de cada vez \`a panela e cozinhe até que as elas ficam adquiram uma cor bronzeada clara. Você vai terminar de fritar mais tarde.
\item Repita o processo até ter pré-fritado todos os pedaços de mandioca.
\end {itemize}
\item {\bf Preparando os molhos}
\begin {itemize}
\item Em um pequeno prato, misture a maionese com ketchup para fazer um molho de rosa.
\item Em um segundo prato pequeno, misture a maionese com a mostarda.
\end {itemize}
\item {\bf Segunda fritura da mandioca}
\begin {itemize}
\item Quando seus convidados estiverem prontos para ser servidos, aque\c{c}a novamente o óleo de canola.
\item Retorne os pedaços de mandioca (6 ou 8 de cada vez) para o óleo quente e termine de fritar até que adquiram uma linda cor bronzeada. Você deve observar atentamente neste ponto, porque eles vão cozinhar rapidamente.
\item Remova a mandioca frita para uma toalha de papel ou uma grelha de metal. 
\item Polvilhe levemente com sal enquanto a mandioca ainda est\'a quente.
\item Sirva como aperitivo com os molhos.
\end {itemize}
\end {enumerate}
\end {description}
\end {document}
