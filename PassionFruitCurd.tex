
\documentclass [11pt, letterpaper] {article}
\input {headings}
\newcommand \recipeName {Curdo de Maracuj\'a}
\chead {\recipeName}

\begin {document}
\input {title}

\begin {flushright}
{\bf Adaptado de America's Test Kitchen Lemon Curd}
\end {flushright}

\begin {description}

\item [Ingredientes:] \ \\
\begin {itemize}
\item 1/3 xícara de suco de maracujá
\item 2 ovos grandes
\item 1 gema de ovo
\item 1/2 xícara de açúcar (3 1/2 onças)
\item 2 colheres de sopa de manteiga sem sal, cortada em cubos de 1/2-inch e gelada
\item 1 colher de sopa de creme de leite
\item 1 pitada de sal
\end {itemize}

\item [Procedimento:] \ \\
\begin {enumerate}
\item {\bf Cozinhe o curdo}
\begin {itemize}
\item Aqueça o suco de maracujá em uma panela não reativa em fogo médio até que esteja quente, mas não fervendo. 
\item Na tigela de uma batedeira, bata os ovos e a gema; gradualmente adicione o açúcar.
\item Adicione lentamente o suco de maracujá aquecido aos ovos.
\item Despeje os ovos batidos com o suco na panela ainda quente.
\item Cozinhe em fogo médio, mexendo constantemente com uma colher de pau até que a mistura registre 75 graus Celcius  no termômetro de leitura instantânea e seja suficientemente grossa para grudar na colher, cerca de 3 minutos.
\end {itemize}
\item {\bf Adicione os ingredientes frios e coe}
\begin {itemize}
\item Retire a panela do fogo e mexa com manteiga gelada até incorporar.
\item Misture o creme de leite e o sal.
\item Passe por um coador de malha fina sobre uma tigela não reativa.
\item Cubra a superfície diretamente com plástico; refrigere até ser necessário.
\end {itemize}
\end {enumerate}

\end {description}
\end {document}
