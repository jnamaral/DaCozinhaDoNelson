\documentclass [11pt, letterpaper] {article}
\input {headings}
\newcommand \recipeName {Moqueca de Peixe Bahiana}
\chead {\recipeName}

\begin {document}
\input {title}

Moqueca é um prato tradicional da costa nordestina do Brasil. Consiste em
um peixe cozido com leite de c\^oco, óleo de dend\^e e pimenta de cheiro.
O óleo de dend\^e é essencial para o sabor distinto da Moqueca. A \'oleo de dend\^e africano,
 mais encontrado em lojas étnicas fora do Brasil, tem um
sabor mais forte que o do Brasil. Assim, se estiver usando esse óleo,
reduza a quantia para 2 colheres de chá (um no início e um no
fim). Prefiro usar uma pequena pimenta do Brasil chamada `` Pimenta
de Cheiro'', literalmente 'pimenta fragante'. Pode ser substituído pela
pimentas preservadas tipicamente encontrados em lojas chinesas fora do Brasil.

\vspace {0.3in}

\begin {description}

\item [Ingredientes:] \ \\
\begin {itemize}
\item 2 lb de filetes de peixes brancos de carne firme (congrio rosado)
\item 1 lb de camarão com casca
\item 1 8 oz pode de suco de molusco
\item 1 de leite de coco não açucarado
\item 1 pimentão vermelho
\item 1 pimentão verde
\item 1 cebola grande
\item 1 xícara de tomates enlatados, semeados e em cubos em cubos
\item 2 colheres de sopa de azeite de dend \^ {e} (óleo de palma vermelho)
\item 3 colheres de sopa de óleo vegetal
\item 1 pimenta de cheiro (mais se preferir mais apimentado)
\item Suco de 1 limão
\item Cilantro
\item Sal e pimenta preta
\end {itemize}

\item [Procedimento:] \ \\
\begin {enumerate}
\item {\bf Marinate the Fish and the shrimp}
\begin {itemize}
\item Corte o peixe em cubos de aproximadamente 2 polegadas.
\item Descasque o camarão, reservando as cascas.
\item Marinate o peixe e o camarão, em tigelas separadas (30 minutos a 1 hora) com sal, pimenta preta,
 suco de limão, duas colheres de sopa de cebola finamente picada e uma colher de sopa
 de coentro finamente picado.
\end {itemize}
\item {\bf Prepare o caldo de camarão}
\begin {itemize}
\item Adicionar uma colher de sopa de óleo vegetal a uma frigideira quente
 e salteie as cascas de camarão por um minuto, adicione 1 xícara de água e deixe cozinhar por 3 minutos.
\item Use um coador para coar o caldo de camarão.
\item Misture o suco de moluscos no caldo de camarão e reserve.
\end {itemize}
\item {\bf Prepare a Base Moqueca}
\begin {itemize}
\item Descasque e remova as sementes dos piment\~oes e corte em tiras.
\item Corte as cebolas em rodelas finas.
\item Usando uma panela de base pesada aquecida, adicione 2 colheres de sopa de óleo vegetal
 e 1 colher de sopa de azeite de dend\^e.
\item Salteie os pimentões e as cebolas até eles estiverem amolecidos.
\item Adicione a pimenta de cheiro, o sal e a pimenta preta.
\item Adicione os tomates em cubos.
\item Adicione a mistura de suco de camarão e suco de moluscos e deixe esfriar até o líquido reduzir a metade.
\end {itemize}
\item {\bf Termine a Moqueca}
\begin {itemize}
\item Coleque o leite de c\^oco em um recipiente que possa ir ao microondas e aque\c{c}a no microondas at\'e estar aquecido mas n\~ao estar fervendo.
\item Em fogo alto, deixe a base do moqueca cozinhar ate' estar borbulhando bastante. 
\item Desligue o fogo e mova a panela para um queimador frio do fog\~ao.
\item Usando uma colher furada, levante o peixe do líquido de tempero e adicione \`a panela.
\item Adicione o camarão descascado.
\item Cubra a panela e deixe coberta por 2 minutes (ou mais se necessario) ate' o camar\~ao n\~ao estar mais transparente e o peixe estar cozido ao ponto.
\item Adicione o leite de c\^oco aquecido.
\item Adicione 2 colheres de sopa de coentro verde picado.
\item Adicione 1 colher de sopa de azeite de dend\^ e.
\item Transfira o moqueca para um prato de servir quente.
\item Sirva com arroz branco.
\end {itemize}
\end {enumerate}
\end {description}
\end {document}
