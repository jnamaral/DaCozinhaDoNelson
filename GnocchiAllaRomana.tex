
\documentclass [11pt, letterpaper] {article}
\input {headings}
\newcommand \recipeName {Gnocchi alla Romana}
\newcommand \fileName {GnocchiAllaRomana}
\chead {\recipeName}

\begin {document}
\input {title}

\begin {flushright}
{De Mario Batali.}
\end {flushright}

Estávamos em Roma de férias, as nossas \'ultimas férias de casal antes que o Daniel chegasse em 2004. Nós fomos em um tour de comida de Roma e nos encontramos com uma crítica de restaurantes do New Yorker morando em Roma. Ela nos deu uma lista de pequenos restaurantes difíceis de encontrar que dever\'iamos visitar. Em um destes restaurantes em um beco estreito, tentei pela primeira vez esse prato. Fiquei enamorado imediatamente. De volta para casa procurei e achei esta receita de Mario Batali. Funciona muito bem. É um daqueles pratos que te bastante gordura mas parecem ser "leves".

\begin {description}

\item [Ingredientes:] \ \\
\begin {itemize}
\item 3 xícaras de leite integral
\item 1 colher de chá de sal
\item 6 colheres de sopa de manteiga e 2 colheres de sopa
\item 1 xícara de semolina
\item 1/2 xícara de queijo Parmigianno-Reggiano ralado mais 1/2 xícara
\item 4 gemas de ovo
\end {itemize}

\item [Procedimento:] \ \\

\begin {enumerate}
\item {\bf Prepare Surface}
\begin {itemize}
\item Limpe uma área em uma bancada ou use uma assadeira.
\item Spray com spray para cozinhar.
\end {itemize}

\item {\bf Cook Gnocchi}
\begin {itemize}
\item Em uma grande panela não reativa aquecer o leite, manteiga e sal.
\item Adicione a semolina em um fluxo fino e constante enquanto bateu vigorosamente com um batedor de arame.
\item À medida que a mistura engrossa, mude para uma colher de madeira plana.
\item Cozinhe até que a mistura é engrossada e começa a se afrouxar do fundo da panela.
\end {itemize}

\item {\bf Incorporate Yolks and Parmesan}
\begin {itemize}
\item Remover do calor.
\item Incorporar gemas de ovos misturando vigorosamente.
\item Incorporar 1/2 xícara de parmesão ralado.
\item Despeje na superfície preparada e espalhe até 1/2 polegada de espessura.
\item Permitir arrefecer.
\end {itemize}

\item {\bf Baking the Nhochchi}
\begin {itemize}
\item Pré-aqueça o forno para 425 F.
\item Grease um prato de cozimento com manteiga.
\item Corte o Gnnochi em pequenos quadrados.
\item Organize os quadrados na assadeira.
\item Polvilhe com o restante 1/2 xícara de queijo parmesão.
\item Aque\c{c}a até a parte de cima ficar levemente dourada.
\item Sirva imediatamente.
\end {itemize}

\end {enumerate}
\end {description}
\input{\imageDir/\fileName/imageTable}
\end {document}