
\documentclass [11pt, letterpaper] {article}
\input {headings}
\newcommand \recipeName {Pêra Assada}
\chead {\recipeName}

\begin {document}
\input {title}

\begin {flushright}
{\bf From {\it Lidia's Italian-American Kitchen} por Lidia M. Batianich}
\end {flushright}

\begin {description}

\item [Ingredientes:] \ \\
\begin {itemize}
\item 8 p\^eras Bosch maduras mas ainda firmes
\item 4 xícaras de uvas vermelhas sem sementes
\item 1 xícara de açúcar
\item 1 1/3 xícara de Moscato ou outro vinho branco frutado
\item Suco de 2 limões
\item 2 colheres de sopa de geléia de damasco
\item 1 colher de sopa de extrato de baunilha
\end {itemize}

\item [Procedimento:] \ \\
\begin {enumerate}
\item {\bf Pré-aqueça o forno para 180 C}
\item {\bf Prepare uma cama de uvas}
\begin {itemize}
\item Remova as uvas dos cachos.
        \item Lave as uvas bem sob água fria, esfregando-as suavemente e drenando bem. Esta lavagem \'e importante para remover o fermento natural (parece um p\'o branco) da casca da uva. Depois de lavadas as uvas v\~ao estar mais escuras.
\item Coloque as uvas em uma assadeira.
\end {itemize}
\item {\bf Misture os sabores}
\begin {itemize}
\item Misture o açúcar, Moscato, suco de um limão, geléia de damasco e baunilha em um prato.
\item Derrame a mistura sobre as uvas.
\end {itemize}
\item {\bf Preparar água acidulada}
\begin {itemize}
\item Misture o suco de um limão em uma grande tigela de água.
\end {itemize}
\item {\bf Prepare as peras}
\begin {itemize}
\item Lave as peras, drene bem.
\item Corte cada pera em quatro partes no sentido longitudinal.
\item Remova o centro, sementes e hastes.
\item A medida que cada peda\c{c}o de p\^era est\'a pronto, coloque  na água acidulada para evitar que oxide e fique escuro.
\end {itemize}
\item {\bf Asse as peras}
\begin {itemize}
\item Remova os peda\c{c}os de p\^era da água acidulada, escorra bem e coloque-os na cama de uvas com o  corte para cima.
\item Assar, descoberto, até que as p\^eras estejam macias e o líquido ao redor das uvas seja uma calda grossa (cerca de 50 minutos a 1 hora).
\end {itemize}
\item {\bf Reduza a calda}
\begin {itemize}
\item Remova as p\^eras e as uvas do prato com uma colher furada.
\item Cozinhe a calda no fogão até obter uma calda um pouco grossa.
\end {itemize}
\item {\bf Serving}
\begin {itemize}
\item Sirva as p\^eras com uvas ainda quentes com uma colher de creme de leite batido aromatizado com baunilha.
\end {itemize}
\end {enumerate}
\end {description}
\end {document}
