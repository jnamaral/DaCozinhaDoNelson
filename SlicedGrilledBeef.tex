\documentclass [11pt, letterpaper] {article}
\input {headings}
\newcommand \recipeName {Picanha Fatiada}
\newcommand \fileName {SlicedGrilledBeef}
\chead {\recipeName}

\begin {document}
\input {title}

Para os meus amigos churrasqueiros do mundo todo. Aqui est\'a uma receita de um churrasco que pode ser completamente preparado antes do primeiro convidado chegar. \'E ideal para uma noite mais requintada quando voc\^e quer ter mais tempo para conversar com os seus convidados no lugar de se preocupar com a churrasqueira.

\begin{description}

\item[Ingredientes:]\ \\
	\begin{itemize}
	\item Picanha (com uma camada fina (0.5 cm) de gordura)
	\item Molho de Soja
	\item Sal
	\item Mostarda estilo Dijon 
	\item Pimenta preta mo\'{i}da na hora 
	\item Oleo de cozinhar sem sabor (Canola, Milho, ou Soja) 
	\item Azeite de Oliva Extra Virgin
	\item Manjeric\~ao Fresco 
	\end{itemize}

\item[Procedimento:]\ \\
	\begin{enumerate}
	\item {\bf Tempere a picanha (pelo menos uma hora antes de assar mas at\'e um dia antes.)}
	\begin{itemize}
	\item Seque bem a picanha com toalhas de papel.
	\item Polvilhe muito levemente com sal fino (o molho de soja e a mostarda tamb\'em s\~ao salgados).
	\item Esfregue molho de soja por todos os lados da picanha.
	\item Polvilhe com pimenta preta mo\'{\i}da na hora por todos os lados.
	\item Passe uma camada fina de mostarda por todos os lados da picanha.
	\item Deixe descan\c{c}ar por pelo menos uma hora (a temperatura ambiente) ou por at\'e 24 horas (no refrigerador). Se refrigerar, retire do refrigerador umas duas horas antes de assar.
	\end{itemize}
	\item {\bf Assando a picanha (instru\c{c}\~oes para a grelha, adapte para churrasqueira ou para uma frigideira de ferro e um forno quente)}
	\begin{itemize}
	\item Deixe a sua grelha o t\~ao quente quanto poss\'ivel (a minha chega a 275C) 
	\item Limpe bem a grelha com uma escova de a\c{c}o.
	 \item Coloque uma pequena quantidade de \'oleo com alto ponto de queimada (\'oleo de canola, de milho, ou de soja) em uma vasilha pequena, dobre um peda\c{c}o de toalha de papel v\'arias vezes. Segure o papel com pegadores, mergulhe o papel no oleo e passe em cima da grelha. Cubra a grelha e deixe o oleo queimar por um minuto. Repita o processo tr\^es ou quatro vezes para reduzir a possibilidade da carne grudar na grelha.
	 \item Asse a picanha, deixando o primeiro lado por mais tempo do que o segundo.
	 \item A medida que cada peda\c{c}o fique pronto (a picanha tem que estar mal passada --- a temperatura interna deve estar entre 125F e 130F em um termometro de leitura instant\^anea), remova o peda\c{c}o e coloque em uma vasilha coberta (uma tigela coberta com um prato ou uma panela pesada com tampa).
	 \item Mantenha a picanha assada coberta at\'e que ela esfrie o suficiente para manusear (20 ou 30 minutos) ela vai continuar cozinhando por este tempo. A picanha tamb\'em vai liberar muito suco. Guarde o suco.
	\end{itemize}
	\item {\bf Fatiando a picanha para servir}
	\begin{itemize}
        		\item Selecione uma t\'abua de cortar da qual seja f\'acil coletar os sucos da picanha. A medida que voce fatia, coloque os sucos e a picanha fatiada de volta na vasilha coberta.
		\item Usando uma faca muito afiada, fatie a picanha em sentido contr\'ario \`a fibra da carne em fatias muito finas. 
		\item Adicione algumas colheres de azeite de oliva extra virgin \`a picanha fatiada.
		\item Perto do momento de servir, pique o manjeric\~ao em tirinhas bem fininhas, adicione \`a picanha fatiada e misture muito bem (o melhor \`e usar as m\~aos para misturar, principalmente se tiver fazendo uma quantidade grande de picanha). 
		\item Sirva \`a temperatura ambiente ou ainda morna.
	\end{itemize}
     	\end{enumerate}         
\end{description}
\input{\imageDir/\fileName/imageTable}
\end{document}



