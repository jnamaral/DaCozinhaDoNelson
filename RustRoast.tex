
\documentclass [11pt, letterpaper] {article}
\input {headings}
\newcommand \recipeName {Carne com Molho Ferrugem}
\newcommand \fileName {RustRoast}
\chead {\recipeName}

\begin {document}
\input {title}

Este assado tem poucos ingredientes para lhe dar sabores. O rico sabor depende de um intenso e cuidadoso preparo da carne e dos tomates. Esta é a minha versão de um prato conhecido no Brasil como ``Carne the Panela". Este prato também é encontrado em restaurantes mais exclusivos como "Carne com Molho Ferrugem". A palavra ferrugem refere-se à cor da carne e do molho devido ao cuidadoso douramento da carne e dos tomates. Eu gosto de usar uma panela de ferro fundido de fundo largo para fritar a carne e os tomates. O ferro fundido é a panela original para esta receita no Brasil e produz o melhor resultado em termos de cor e sabor. É também o equipamento mais fácil de usar porque causa a quantidade certa de aderência ao pote para produzir o molho de ferrugem desejado. Você também pode usar uma panela de aço inoxidável de fundo pesado. Descobri que uma panela de ferro esmaltada não é o melhor para dourar a carne e que as panelas de alumínio tamb\'em podem
ficar muito quentes muito rapidamente. No entanto, eu gosto de transferir a carne com molho para um panela de ferro esmaltada para cozinhar a carne no forno. Resista à tentação de adicionar pimenta preta, folhas de louro ou quaisquer outros aromas tradicionais de carnes assadas no forno. Você está buscando a pureza do casamento de carne e tomate e você quer mostrar sua técnica neste prato, criando os sabores na forma como você cozinha esses dois ingredientes. Este prato exige uma quantidade razoável de tempo e atenção para garantir que o douramento da carne e dos tomates seja feito corretamente. Também precisa de algum tempo no forno. É um prato ideal para o jantar pois pode ser feito com anteced\^encia. O produto final vale a dedicação de tempo.
 
\begin {description}

\item [Ingredientes:] \ \\
\begin {itemize}
\item Dois quilos de acem
\item Molho de soja
\item Sal
\item Mostarda estilo Dijon
\item Óleo de cozinha sem sabor, como óleo de canola
\item Extrato de tomate
\item Lata de tomates inteiros
\item Um pacote de pó de gelatina sem aroma
\item Uma colher de sopa de amido de milho
\end {itemize}


\item [Procedimento:] \ \\
\begin {enumerate}
\item {\bf Corte e salgue a carne (pelo menos uma hora antes de começar a cozinhar e até um dia antes).}
\begin {itemize}
\item Corte a carne em pedaços grandes (cerca de 3x2x2 polegadas) garantindo que você corte a fibra muscular.
\item Polvilhe muito levemente com sal em todos os lados.
\item Passe molho de soyo em todos os lados de cada pedaço de carne - a quantidade de molho de soja depende do quanto é salgado o seu molho de soja.
\item Espalhe um casaco leve de mostarda Dijon preparado na carne (cerca de 1 colher de sopa por 3 quilos de satisfação). Você não quer um sabor distinto de mostarda no produto final. Um leve toque de mostarda aumenta o sabor da carne.
\item Deixe descansar durante pelo menos uma hora (à temperatura ambiente) e até 24 horas (na geladeira). Se for refrigerado, você pode tirá-lo da geladeira um par de horas antes de começar a cozinhar.
\end {itemize}
\item {\bf Doure a carne}
\begin {itemize}
\item Aqueça uma panela de ferro fundido com fundo grande em fogo médio-alto até ficar muito quente, mas não estar levantando fuma\c{c}a.
\item Coloque óleo suficiente para revestir o fundo da panela. Mexa a panela para o \'oleo cobrir cerca de um cent\'imetro ao redor dos lados para evitar queimar.
\item Coloque pedaços de carne na panela de maneira que eles não estejam tocando uns nos outros ou os lados da panela - você vai trabalhar em bateladas. Você pode cobrir a panela para reduzir o espalhamento de gordura no fog\~ao deixando um pequeno espaço para o vapor escapar.
\item Quando os peda\c{c}os iniciais ficaram cinzas na parte superior e o líquido tiver evaporado, gire todos os peda\c{c}os para dourar o outro lado. Você pode adicionar mais alguns pedaços de carne à panela porque os originais encolheram.
\item Mantenha um olho em seu fogo. Se estiver muito quente a carne vai fritar rápido demais com o risco de desenvolver compostos de sabor amargo. Se o fogo estiver muito baixo, a fritura da carne levará muito tempo. Você quer que a carne torne a cor da ferrugem, mas que não fique preta. Em qualquer ponto, se a carne come\c{c}ar a ficar muito escura, adicione um pouco de água à panela e esfregue os pedaços de carne nas paredes e no fundo da panela para remover o que estiver grudado na panela - você também pode usar uma colher de madeira plana para raspar os pedaços dourados no molho. Sempre que adicionar água ou carne crua à panela, você deve ter um pouco de molho de cor de ferrugem. Mova os pedaços de carne para que eles absorvam esse molho.
\item Você pode deixar os pedaços de carne que já estão dourados na panela quando você adiciona carne crua. Este processo de escurecimento e adição de líquido para liberar os pedaços dourados da panela é a técnica para criar o molho de ferrugem.
\item Se a panela estiver cheia ou se alguns pedaços de carne começam a ficar muito escuros e secos, remova esses pedaços da panela de fritura para um panela de ferro esmaltado limpa.
\item Enquanto a carne está dourando, prepare os tomates (veja abaixo) e abra a lata de massa de tomate.
\item Uma vez que todos os pedaços de carne são dourados, remova toda a carne para a panela de ferro esmaltado.
\end {itemize}

\item {\bf Toste os Tomates}
\begin {itemize}
        \item Coloque um grande coador de furos grossos sobre uma tigela grande.
\item Abra a lata de tomates e despeje seu conteúdo no filtro.
\item Usando uma tesoura de cozinha (ou uma pequena faca ou espátula), corte cada tomate inteiro pela metade.
\item Gire suavemente os tomates inteiros sobre o coador com uma espátula de borracha para escorrer o suco, mas não misture os tomates - mantenha os tomates tão inteiros quanto você conseguir.
\item Uma vez que toda a carne esteja dourada e removida da panela, adicione os tomates escorridos, sem o líquido, à panela quente.
\item Agite os tomates ao redor para ajudar a remover todos os pedaços dourados que estão presos na panela.
\item Polvilhe o pó de gelatina sobre o líquido de tomate que está agora na tigela.
\item Deixe os tomates cozinhar na gordura até que estejam bastante secos e eles também começam a adquirir uma cor de ferrugem.
\item Você precisa cuidar com bastante atenção porque os tomates têm uma quantidade significativa de açúcar. Demora tempo para o líquido evaporar. Mas, uma vez que todo o líquido evaporou eles podem queimar muito rapidamente. Você quer chegar ao estágio do tomate tostado, mas não ao estágio queimado.
\item Se necessário, a qualquer momento, você pode adicionar algumas gotas de água para evitar que os tomates queimem, mas apenas faça isso se necessário, pois prolongará seu tempo de cozimento.
\item Quando os tomates estiverem tostados, despeje toda a pequena lata de pasta de tomate na panela e mexa com uma colher de madeira plana até que a massa de tomate se torne mais seca e adquira uma cor um pouco mais escura.
\item Misture o líquido de tomate com a gelatina na tigela e despeje-o na panela mexendo por aí.
\end {itemize}

\item {\bf Braise the beef}
\begin {itemize}
\item Pré-aqueça o forno a 150 C.
\item Despeje todo o conteúdo da panela que contém os tomates na panela de ferro esmaltada que tem a carne dourada.
\item Enxágüe a panela de tomate com água e adicione a água na panela de ferro esmaltada. Você quer líquido suficiente para cobrir toda a carne.
\item Cubra na panela de ferro esmaltada e transfira para o forno.
\item Deixe cozinhar de duas a três horas até que a carne esteja muito macia.
\item Você pode ajustar o forno para assar por duas horas e deixar a panela no forno depois disso. Assim, este passo pode ser feito durante a noite, ou enquanto você sai, se você tiver um forno com um recurso de temporizador.
\end {itemize}
\item {\bf Finalizando o cozimento}
\begin {itemize}
\item Retire a panela do forno e coloque-a numa posição levemente inclinada. Por exemplo, coloque um lado da panela sobre duas par de tábuas de cortar de madeira e o outro lado no balcão.
\item Usando uma colher grande, recolha tanta gordura superficial quanto puder. Essa gordura será descartada.
\item Misture o amido de milho com uma x\'icara de água à temperatura ambiente. Para evitar a forma\c{c}\~ao de caroços é mais fácil adicionar primeiro duas colheres de água ao amido e dissolvê-lo antes de adicionar o restante da água.
\item Despeje a mistura de amido de milho e água sobre a carne.
\item Usando uma espátula de borracha, esfregue suavemente toda a circunferência da panela para fazer com que todos os pedaços dourados que se formaram ao redor do pote caiam novamente no molho. Em seguida, mexa suavemente.
\item Prove o sal.
\item Re-aqueça a carne suavemente no forno ou com uma chama baixa cuidando para que não queime no fundo.
\end {itemize}
     \end {enumerate}
\end {description}
\input{\imageDir/\fileName/imageTable}
\end {document}
