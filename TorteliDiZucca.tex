\documentclass [11pt, letterpaper] {article}
\input {headings}
\newcommand \recipeName {Tort\'eis de Moranga}
\chead {\recipeName}

\begin {document}
\input {title}

\begin {flushright}
Adaptado de Lidia Bastianich.
\end {flushright}

Esta receita vem  daa região de Reggio Emilia da Itália, que está localizada ao leste de Parma e a oeste de Modena. Tortelli di zucca alla Mantovana, é um delicado tortelli recheado com um purê de abóbora assada que é aromatizado com Mostarda di Cremona, uma fruta em calda doce e picante. Cremona está ao Norte da região de Reggio Emilia. Nota:
Em Edmonton, você encontra a moranga (ou abóbora de Mantovana) no Superstore, no T \& T ou no Lucky 97. Eu uso manteiga e leite na massa para fazer uma massa macia que é mais adequada ao recheio da abóbora delicada. A Mostarda di Cremona deve ser encomendada ou adquirida em uma loja italiana em outro lugar. É importante preparar tanto a massa quanto o recheio várias horas antes de formar os tort\'eis.

\begin {description}

\item [Ingredientes:] \ \\
\begin {itemize}
\item 6 ovos
\item 3 colheres de sopa de manteiga sem sal derretida
\item 3 colheres de sopa de leite integral
\item 1/2 colher de chá de sal
\item 4 xícaras de farinha
\item 1 quilo de moranga
\item 1 xícara de queijo fresco Parmigiano-Reggiano ralado
\item 1/2 xícara  de pedaços mostarda di Cremona finamente cortada
\item 8 biscoitos amaretti, esmagados em migalhas finas (cerca de 2/3 xícaras)
\item Noz moscada recentemente ralada
\item Pimenta preta recém-moída
\item 2 gemas de ovo grandes
\item 5 a 7 colheres de sopa de manteiga sem sal
\item Folhas de sálvia fresca
\item 2/3 a 1 xícara de queijo Parmigiano-Reggiano ralado
\item 1 colher de sopa de sal
\end {itemize}


\item [Procedimento:] \ \\

\begin {enumerate}
\item {\bf Prepare a massa}
\begin {itemize}
\item Quebre os ovos em uma tigela grande e misture com um garfo para quebrar as gemas.
\item Ao misturar os ovos, derrame a manteiga derretida em um fluxo lento.
\item Adicione o leite e o sal e misture.
\item Comece lentamente a incorporar a farinha até que você não consiga mais misturar com o garfo.
\item Adicione um pouco mais de farinha e comece a incorporá-la com a mão.
\item Coloque a massa em uma superfície plana e continue amassando e incorporando farinha até que a massa esteja ligeiramente firme e bastante lisa.
\item Pulverize a tigela com spray de cozinha, coloque a bola de massa na tigela e cubra com plástico
\item Deixe descansar por pelo menos por uma hora à temperatura ambiente.
\item Coloque a massa em superfície plana e amasse novamente até ficar muito lisa.
\item Coloque a bola de massa de volta na tigela pulverizada e cubra com uma embalagem de plástico para descansar por pelo menos uma hora ou até a hora de formar o raviolli.
\end {itemize}

\item {\bf Make the Squash Puree}
\begin {itemize}
\item Pré-aqueça o forno a 375 F.
\item Lave a moranga.
\item Corte em quartos e depois corte cada quarto pela metade transversalmente.
\item Retire as sementes com uma colher.
\item Arranje os pedaços de abóbora em uma forma de assar com a parte cortada para cima.
\item Asse até que esteja macia e seja f\'acil de perfurar com uma faca de ponta, entre 40 min. e uma hora.
\item Deixe a moranga esfriar.
\item Raspe a moranga na tigela de processador de alimentos.
\item Processo até formar uma massa bem homog\^enea.
\item Transfira para uma tigela e refrigere por pelo menos 30 minutos.
\end {itemize}

\item {\bf Misture o recheio}
\begin {itemize}
\item Adicione a mostarda cortada, queijo ralado e amaretti esmagado.
\item Tempere com noz-moscada, pimenta e sal a gosto.
\item Adicione as gemas de ovo e misture até ficarem lisas e bem misturadas.
\end {itemize}


\item {\bf Forme os tort\'eis}
\begin {itemize}
\item Cubra uma assadeira com papel de pergaminho.
\item Corte 1/3 da massa e mantenha a massa restante coberta com a película plástica.
\item Abra 1/3 da massa com um rolo formando uma forma retangular até que a massa seja bastante fina. Use uma quantidade abundante de farinha ao rolar.
\item Limpe o lado superior da massa com um pincel de massa para remover todo o excesso de farinha.
\item Largue cerca de meia colher de sopa de recheio ao longo de um lado da massa em intervalos regulares, deixando bastante margem para permitir que o final da massa se dobre sobre os montes de recheio.
\item Usando o dedo e um pequeno prato com água fria, coloque água ao redor de cada montinho de recheio.
\item Seque bem suas mãos, dobre a margem da massa sobre os montes de recheio e pressione firmemente em torno de cada monte.
\item Usando uma carretilha de massa, corte os tort\'eis quadrados em torno de cada monte de recheio.
\item Coloque os tort\'eis bem espa\c{c}ados um do outro em formas revestidas com pergaminho.
\item Repita o processo até que toda a massa esticada seja usada.
\item Repita com o restante a massa, esticando 1/3 de cada vez.
\item Se não estiver cozinhando em breve, depois de um tempo, vire os tort\'eis de lado na forma revestida com pergaminho para que não grudem no papel.
\end {itemize}

\item {\bf Cozinhe o Raviolli e Faça o Molho}
\begin {itemize}
\item Coloque uma grande panela de água para ferver. Adicione uma colher de sopa de sal.
\item Ligue o forno a 100 C e coloque os pratos de servir no forno.
\item Corte folhas de sálvia em tiras finas --- é chamado de uma chiffonada.
\item Largue o raviolli na água fervente.
\item Coloque a manteiga em uma grande frigideira que n\~ao grude e coloque em fogo médio.
\item Quando a manteiga derreter e começa a espumar, coloque a chiffonada de sálvia na manteiga e mexa. A s\'alvia ficar crocante  em menos de um minuto.
\item Quando a s\'alvia est\'a crocante e a manteiga ficou dourada, se o raviolli ainda não estiver pronto, adicione algumas colheres de sopa da água da panela onde os tort\'eis est\~ao cozinhando à manteiga para diminuir o cozimento e diminuir o calor.
\item Os tort\'eis deve ser cozido cerca de um minuto depois de flutuar \`a superfície. O tempo de cozedura total será entre 4 e 5 minutos.
\item Usando uma colher de malha, retire os tort\'eis da água fervente e coloque-os diretamente na panela com manteiga e sálvia.
\item Frite por alguns minutos agitando a panela com freqüência.
\item Adicione um pouquinho de água quente da panela em que cozinhou os tort\'eis para fazer uma pequena quantidade de um molho grosso.
\item Sirva imediatamente em platos aquecidas.
\item Polvilhe o Parmigianno-Regiano recentemente ralado em cada prato.
\item Polvilhe uma pequena quantidade de pimenta recém-ralada.
\end {itemize}

\end {enumerate}
\end {description}
\end {document}
